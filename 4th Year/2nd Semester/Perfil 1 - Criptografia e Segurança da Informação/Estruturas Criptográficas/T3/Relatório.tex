\documentclass[12pt]{report}
\renewcommand{\thesection}{\arabic{section}}
\usepackage[portuguese]{babel}

    \usepackage[breakable]{tcolorbox}
    \usepackage{parskip} % Stop auto-indenting (to mimic markdown behaviour)
    
    \usepackage{iftex}
    \ifPDFTeX
    	\usepackage[T1]{fontenc}
    	\usepackage{mathpazo}
    \else
    	\usepackage{fontspec}
    \fi

    % Basic figure setup, for now with no caption control since it's done
    % automatically by Pandoc (which extracts ![](path) syntax from Markdown).
    \usepackage{graphicx}
    % Maintain compatibility with old templates. Remove in nbconvert 6.0
    \let\Oldincludegraphics\includegraphics
    % Ensure that by default, figures have no caption (until we provide a
    % proper Figure object with a Caption API and a way to capture that
    % in the conversion process - todo).
    \usepackage{caption}
    \DeclareCaptionFormat{nocaption}{}
    \captionsetup{format=nocaption,aboveskip=0pt,belowskip=0pt}

    \usepackage[Export]{adjustbox} % Used to constrain images to a maximum size
    \adjustboxset{max size={0.9\linewidth}{0.9\paperheight}}
    \usepackage{float}
    \floatplacement{figure}{H} % forces figures to be placed at the correct location
    \usepackage{xcolor} % Allow colors to be defined
    \usepackage{enumerate} % Needed for markdown enumerations to work
    \usepackage{geometry} % Used to adjust the document margins
    \usepackage{amsmath} % Equations
    \usepackage{amssymb} % Equations
    \usepackage[mathletters]{ucs} % Extended unicode (utf-8) support
    \usepackage[utf8x]{inputenc} % Allow utf-8 characters in the tex document
    \usepackage{textcomp} % defines textquotesingle
    % Hack from http://tex.stackexchange.com/a/47451/13684:
    \AtBeginDocument{%
        \def\PYZsq{\textquotesingle}% Upright quotes in Pygmentized code
    }
    \usepackage{upquote} % Upright quotes for verbatim code
    \usepackage{eurosym} % defines \euro
    \usepackage[mathletters]{ucs} % Extended unicode (utf-8) support
    \usepackage{fancyvrb} % verbatim replacement that allows latex
    \usepackage{grffile} % extends the file name processing of package graphics 
                         % to support a larger range
    \makeatletter % fix for grffile with XeLaTeX
    \def\Gread@@xetex#1{%
      \IfFileExists{"\Gin@base".bb}%
      {\Gread@eps{\Gin@base.bb}}%
      {\Gread@@xetex@aux#1}%
    }
    \makeatother

    % The hyperref package gives us a pdf with properly built
    % internal navigation ('pdf bookmarks' for the table of contents,
    % internal cross-reference links, web links for URLs, etc.)
    \usepackage{hyperref}
    % The default LaTeX title has an obnoxious amount of whitespace. By default,
    % titling removes some of it. It also provides customization options.
    \usepackage{titling}
    \usepackage{longtable} % longtable support required by pandoc >1.10
    \usepackage{booktabs}  % table support for pandoc > 1.12.2
    \usepackage[inline]{enumitem} % IRkernel/repr support (it uses the enumerate* environment)
    \usepackage[normalem]{ulem} % ulem is needed to support strikethroughs (\sout)
                                % normalem makes italics be italics, not underlines
    \usepackage{mathrsfs}
    

    
    % Colors for the hyperref package
    \definecolor{urlcolor}{rgb}{0,.145,.698}
    \definecolor{linkcolor}{rgb}{.71,0.21,0.01}
    \definecolor{citecolor}{rgb}{.12,.54,.11}

    % ANSI colors
    \definecolor{ansi-black}{HTML}{3E424D}
    \definecolor{ansi-black-intense}{HTML}{282C36}
    \definecolor{ansi-red}{HTML}{E75C58}
    \definecolor{ansi-red-intense}{HTML}{B22B31}
    \definecolor{ansi-green}{HTML}{00A250}
    \definecolor{ansi-green-intense}{HTML}{007427}
    \definecolor{ansi-yellow}{HTML}{DDB62B}
    \definecolor{ansi-yellow-intense}{HTML}{B27D12}
    \definecolor{ansi-blue}{HTML}{208FFB}
    \definecolor{ansi-blue-intense}{HTML}{0065CA}
    \definecolor{ansi-magenta}{HTML}{D160C4}
    \definecolor{ansi-magenta-intense}{HTML}{A03196}
    \definecolor{ansi-cyan}{HTML}{60C6C8}
    \definecolor{ansi-cyan-intense}{HTML}{258F8F}
    \definecolor{ansi-white}{HTML}{C5C1B4}
    \definecolor{ansi-white-intense}{HTML}{A1A6B2}
    \definecolor{ansi-default-inverse-fg}{HTML}{FFFFFF}
    \definecolor{ansi-default-inverse-bg}{HTML}{000000}

    % commands and environments needed by pandoc snippets
    % extracted from the output of `pandoc -s`
    \providecommand{\tightlist}{%
      \setlength{\itemsep}{0pt}\setlength{\parskip}{0pt}}
    \DefineVerbatimEnvironment{Highlighting}{Verbatim}{commandchars=\\\{\}}
    % Add ',fontsize=\small' for more characters per line
    \newenvironment{Shaded}{}{}
    \newcommand{\KeywordTok}[1]{\textcolor[rgb]{0.00,0.44,0.13}{\textbf{{#1}}}}
    \newcommand{\DataTypeTok}[1]{\textcolor[rgb]{0.56,0.13,0.00}{{#1}}}
    \newcommand{\DecValTok}[1]{\textcolor[rgb]{0.25,0.63,0.44}{{#1}}}
    \newcommand{\BaseNTok}[1]{\textcolor[rgb]{0.25,0.63,0.44}{{#1}}}
    \newcommand{\FloatTok}[1]{\textcolor[rgb]{0.25,0.63,0.44}{{#1}}}
    \newcommand{\CharTok}[1]{\textcolor[rgb]{0.25,0.44,0.63}{{#1}}}
    \newcommand{\StringTok}[1]{\textcolor[rgb]{0.25,0.44,0.63}{{#1}}}
    \newcommand{\CommentTok}[1]{\textcolor[rgb]{0.38,0.63,0.69}{\textit{{#1}}}}
    \newcommand{\OtherTok}[1]{\textcolor[rgb]{0.00,0.44,0.13}{{#1}}}
    \newcommand{\AlertTok}[1]{\textcolor[rgb]{1.00,0.00,0.00}{\textbf{{#1}}}}
    \newcommand{\FunctionTok}[1]{\textcolor[rgb]{0.02,0.16,0.49}{{#1}}}
    \newcommand{\RegionMarkerTok}[1]{{#1}}
    \newcommand{\ErrorTok}[1]{\textcolor[rgb]{1.00,0.00,0.00}{\textbf{{#1}}}}
    \newcommand{\NormalTok}[1]{{#1}}
    
    % Additional commands for more recent versions of Pandoc
    \newcommand{\ConstantTok}[1]{\textcolor[rgb]{0.53,0.00,0.00}{{#1}}}
    \newcommand{\SpecialCharTok}[1]{\textcolor[rgb]{0.25,0.44,0.63}{{#1}}}
    \newcommand{\VerbatimStringTok}[1]{\textcolor[rgb]{0.25,0.44,0.63}{{#1}}}
    \newcommand{\SpecialStringTok}[1]{\textcolor[rgb]{0.73,0.40,0.53}{{#1}}}
    \newcommand{\ImportTok}[1]{{#1}}
    \newcommand{\DocumentationTok}[1]{\textcolor[rgb]{0.73,0.13,0.13}{\textit{{#1}}}}
    \newcommand{\AnnotationTok}[1]{\textcolor[rgb]{0.38,0.63,0.69}{\textbf{\textit{{#1}}}}}
    \newcommand{\CommentVarTok}[1]{\textcolor[rgb]{0.38,0.63,0.69}{\textbf{\textit{{#1}}}}}
    \newcommand{\VariableTok}[1]{\textcolor[rgb]{0.10,0.09,0.49}{{#1}}}
    \newcommand{\ControlFlowTok}[1]{\textcolor[rgb]{0.00,0.44,0.13}{\textbf{{#1}}}}
    \newcommand{\OperatorTok}[1]{\textcolor[rgb]{0.40,0.40,0.40}{{#1}}}
    \newcommand{\BuiltInTok}[1]{{#1}}
    \newcommand{\ExtensionTok}[1]{{#1}}
    \newcommand{\PreprocessorTok}[1]{\textcolor[rgb]{0.74,0.48,0.00}{{#1}}}
    \newcommand{\AttributeTok}[1]{\textcolor[rgb]{0.49,0.56,0.16}{{#1}}}
    \newcommand{\InformationTok}[1]{\textcolor[rgb]{0.38,0.63,0.69}{\textbf{\textit{{#1}}}}}
    \newcommand{\WarningTok}[1]{\textcolor[rgb]{0.38,0.63,0.69}{\textbf{\textit{{#1}}}}}
    
    
    % Define a nice break command that doesn't care if a line doesn't already
    % exist.
    \def\br{\hspace*{\fill} \\* }
    % Math Jax compatibility definitions
    \def\gt{>}
    \def\lt{<}
    \let\Oldtex\TeX
    \let\Oldlatex\LaTeX
    \renewcommand{\TeX}{\textrm{\Oldtex}}
    \renewcommand{\LaTeX}{\textrm{\Oldlatex}}
    % Document parameters
    % Document title
    \title{T1E01}
    
    
    
    
    
% Pygments definitions
\makeatletter
\def\PY@reset{\let\PY@it=\relax \let\PY@bf=\relax%
    \let\PY@ul=\relax \let\PY@tc=\relax%
    \let\PY@bc=\relax \let\PY@ff=\relax}
\def\PY@tok#1{\csname PY@tok@#1\endcsname}
\def\PY@toks#1+{\ifx\relax#1\empty\else%
    \PY@tok{#1}\expandafter\PY@toks\fi}
\def\PY@do#1{\PY@bc{\PY@tc{\PY@ul{%
    \PY@it{\PY@bf{\PY@ff{#1}}}}}}}
\def\PY#1#2{\PY@reset\PY@toks#1+\relax+\PY@do{#2}}

\expandafter\def\csname PY@tok@w\endcsname{\def\PY@tc##1{\textcolor[rgb]{0.73,0.73,0.73}{##1}}}
\expandafter\def\csname PY@tok@c\endcsname{\let\PY@it=\textit\def\PY@tc##1{\textcolor[rgb]{0.25,0.50,0.50}{##1}}}
\expandafter\def\csname PY@tok@cp\endcsname{\def\PY@tc##1{\textcolor[rgb]{0.74,0.48,0.00}{##1}}}
\expandafter\def\csname PY@tok@k\endcsname{\let\PY@bf=\textbf\def\PY@tc##1{\textcolor[rgb]{0.00,0.50,0.00}{##1}}}
\expandafter\def\csname PY@tok@kp\endcsname{\def\PY@tc##1{\textcolor[rgb]{0.00,0.50,0.00}{##1}}}
\expandafter\def\csname PY@tok@kt\endcsname{\def\PY@tc##1{\textcolor[rgb]{0.69,0.00,0.25}{##1}}}
\expandafter\def\csname PY@tok@o\endcsname{\def\PY@tc##1{\textcolor[rgb]{0.40,0.40,0.40}{##1}}}
\expandafter\def\csname PY@tok@ow\endcsname{\let\PY@bf=\textbf\def\PY@tc##1{\textcolor[rgb]{0.67,0.13,1.00}{##1}}}
\expandafter\def\csname PY@tok@nb\endcsname{\def\PY@tc##1{\textcolor[rgb]{0.00,0.50,0.00}{##1}}}
\expandafter\def\csname PY@tok@nf\endcsname{\def\PY@tc##1{\textcolor[rgb]{0.00,0.00,1.00}{##1}}}
\expandafter\def\csname PY@tok@nc\endcsname{\let\PY@bf=\textbf\def\PY@tc##1{\textcolor[rgb]{0.00,0.00,1.00}{##1}}}
\expandafter\def\csname PY@tok@nn\endcsname{\let\PY@bf=\textbf\def\PY@tc##1{\textcolor[rgb]{0.00,0.00,1.00}{##1}}}
\expandafter\def\csname PY@tok@ne\endcsname{\let\PY@bf=\textbf\def\PY@tc##1{\textcolor[rgb]{0.82,0.25,0.23}{##1}}}
\expandafter\def\csname PY@tok@nv\endcsname{\def\PY@tc##1{\textcolor[rgb]{0.10,0.09,0.49}{##1}}}
\expandafter\def\csname PY@tok@no\endcsname{\def\PY@tc##1{\textcolor[rgb]{0.53,0.00,0.00}{##1}}}
\expandafter\def\csname PY@tok@nl\endcsname{\def\PY@tc##1{\textcolor[rgb]{0.63,0.63,0.00}{##1}}}
\expandafter\def\csname PY@tok@ni\endcsname{\let\PY@bf=\textbf\def\PY@tc##1{\textcolor[rgb]{0.60,0.60,0.60}{##1}}}
\expandafter\def\csname PY@tok@na\endcsname{\def\PY@tc##1{\textcolor[rgb]{0.49,0.56,0.16}{##1}}}
\expandafter\def\csname PY@tok@nt\endcsname{\let\PY@bf=\textbf\def\PY@tc##1{\textcolor[rgb]{0.00,0.50,0.00}{##1}}}
\expandafter\def\csname PY@tok@nd\endcsname{\def\PY@tc##1{\textcolor[rgb]{0.67,0.13,1.00}{##1}}}
\expandafter\def\csname PY@tok@s\endcsname{\def\PY@tc##1{\textcolor[rgb]{0.73,0.13,0.13}{##1}}}
\expandafter\def\csname PY@tok@sd\endcsname{\let\PY@it=\textit\def\PY@tc##1{\textcolor[rgb]{0.73,0.13,0.13}{##1}}}
\expandafter\def\csname PY@tok@si\endcsname{\let\PY@bf=\textbf\def\PY@tc##1{\textcolor[rgb]{0.73,0.40,0.53}{##1}}}
\expandafter\def\csname PY@tok@se\endcsname{\let\PY@bf=\textbf\def\PY@tc##1{\textcolor[rgb]{0.73,0.40,0.13}{##1}}}
\expandafter\def\csname PY@tok@sr\endcsname{\def\PY@tc##1{\textcolor[rgb]{0.73,0.40,0.53}{##1}}}
\expandafter\def\csname PY@tok@ss\endcsname{\def\PY@tc##1{\textcolor[rgb]{0.10,0.09,0.49}{##1}}}
\expandafter\def\csname PY@tok@sx\endcsname{\def\PY@tc##1{\textcolor[rgb]{0.00,0.50,0.00}{##1}}}
\expandafter\def\csname PY@tok@m\endcsname{\def\PY@tc##1{\textcolor[rgb]{0.40,0.40,0.40}{##1}}}
\expandafter\def\csname PY@tok@gh\endcsname{\let\PY@bf=\textbf\def\PY@tc##1{\textcolor[rgb]{0.00,0.00,0.50}{##1}}}
\expandafter\def\csname PY@tok@gu\endcsname{\let\PY@bf=\textbf\def\PY@tc##1{\textcolor[rgb]{0.50,0.00,0.50}{##1}}}
\expandafter\def\csname PY@tok@gd\endcsname{\def\PY@tc##1{\textcolor[rgb]{0.63,0.00,0.00}{##1}}}
\expandafter\def\csname PY@tok@gi\endcsname{\def\PY@tc##1{\textcolor[rgb]{0.00,0.63,0.00}{##1}}}
\expandafter\def\csname PY@tok@gr\endcsname{\def\PY@tc##1{\textcolor[rgb]{1.00,0.00,0.00}{##1}}}
\expandafter\def\csname PY@tok@ge\endcsname{\let\PY@it=\textit}
\expandafter\def\csname PY@tok@gs\endcsname{\let\PY@bf=\textbf}
\expandafter\def\csname PY@tok@gp\endcsname{\let\PY@bf=\textbf\def\PY@tc##1{\textcolor[rgb]{0.00,0.00,0.50}{##1}}}
\expandafter\def\csname PY@tok@go\endcsname{\def\PY@tc##1{\textcolor[rgb]{0.53,0.53,0.53}{##1}}}
\expandafter\def\csname PY@tok@gt\endcsname{\def\PY@tc##1{\textcolor[rgb]{0.00,0.27,0.87}{##1}}}
\expandafter\def\csname PY@tok@err\endcsname{\def\PY@bc##1{\setlength{\fboxsep}{0pt}\fcolorbox[rgb]{1.00,0.00,0.00}{1,1,1}{\strut ##1}}}
\expandafter\def\csname PY@tok@kc\endcsname{\let\PY@bf=\textbf\def\PY@tc##1{\textcolor[rgb]{0.00,0.50,0.00}{##1}}}
\expandafter\def\csname PY@tok@kd\endcsname{\let\PY@bf=\textbf\def\PY@tc##1{\textcolor[rgb]{0.00,0.50,0.00}{##1}}}
\expandafter\def\csname PY@tok@kn\endcsname{\let\PY@bf=\textbf\def\PY@tc##1{\textcolor[rgb]{0.00,0.50,0.00}{##1}}}
\expandafter\def\csname PY@tok@kr\endcsname{\let\PY@bf=\textbf\def\PY@tc##1{\textcolor[rgb]{0.00,0.50,0.00}{##1}}}
\expandafter\def\csname PY@tok@bp\endcsname{\def\PY@tc##1{\textcolor[rgb]{0.00,0.50,0.00}{##1}}}
\expandafter\def\csname PY@tok@fm\endcsname{\def\PY@tc##1{\textcolor[rgb]{0.00,0.00,1.00}{##1}}}
\expandafter\def\csname PY@tok@vc\endcsname{\def\PY@tc##1{\textcolor[rgb]{0.10,0.09,0.49}{##1}}}
\expandafter\def\csname PY@tok@vg\endcsname{\def\PY@tc##1{\textcolor[rgb]{0.10,0.09,0.49}{##1}}}
\expandafter\def\csname PY@tok@vi\endcsname{\def\PY@tc##1{\textcolor[rgb]{0.10,0.09,0.49}{##1}}}
\expandafter\def\csname PY@tok@vm\endcsname{\def\PY@tc##1{\textcolor[rgb]{0.10,0.09,0.49}{##1}}}
\expandafter\def\csname PY@tok@sa\endcsname{\def\PY@tc##1{\textcolor[rgb]{0.73,0.13,0.13}{##1}}}
\expandafter\def\csname PY@tok@sb\endcsname{\def\PY@tc##1{\textcolor[rgb]{0.73,0.13,0.13}{##1}}}
\expandafter\def\csname PY@tok@sc\endcsname{\def\PY@tc##1{\textcolor[rgb]{0.73,0.13,0.13}{##1}}}
\expandafter\def\csname PY@tok@dl\endcsname{\def\PY@tc##1{\textcolor[rgb]{0.73,0.13,0.13}{##1}}}
\expandafter\def\csname PY@tok@s2\endcsname{\def\PY@tc##1{\textcolor[rgb]{0.73,0.13,0.13}{##1}}}
\expandafter\def\csname PY@tok@sh\endcsname{\def\PY@tc##1{\textcolor[rgb]{0.73,0.13,0.13}{##1}}}
\expandafter\def\csname PY@tok@s1\endcsname{\def\PY@tc##1{\textcolor[rgb]{0.73,0.13,0.13}{##1}}}
\expandafter\def\csname PY@tok@mb\endcsname{\def\PY@tc##1{\textcolor[rgb]{0.40,0.40,0.40}{##1}}}
\expandafter\def\csname PY@tok@mf\endcsname{\def\PY@tc##1{\textcolor[rgb]{0.40,0.40,0.40}{##1}}}
\expandafter\def\csname PY@tok@mh\endcsname{\def\PY@tc##1{\textcolor[rgb]{0.40,0.40,0.40}{##1}}}
\expandafter\def\csname PY@tok@mi\endcsname{\def\PY@tc##1{\textcolor[rgb]{0.40,0.40,0.40}{##1}}}
\expandafter\def\csname PY@tok@il\endcsname{\def\PY@tc##1{\textcolor[rgb]{0.40,0.40,0.40}{##1}}}
\expandafter\def\csname PY@tok@mo\endcsname{\def\PY@tc##1{\textcolor[rgb]{0.40,0.40,0.40}{##1}}}
\expandafter\def\csname PY@tok@ch\endcsname{\let\PY@it=\textit\def\PY@tc##1{\textcolor[rgb]{0.25,0.50,0.50}{##1}}}
\expandafter\def\csname PY@tok@cm\endcsname{\let\PY@it=\textit\def\PY@tc##1{\textcolor[rgb]{0.25,0.50,0.50}{##1}}}
\expandafter\def\csname PY@tok@cpf\endcsname{\let\PY@it=\textit\def\PY@tc##1{\textcolor[rgb]{0.25,0.50,0.50}{##1}}}
\expandafter\def\csname PY@tok@c1\endcsname{\let\PY@it=\textit\def\PY@tc##1{\textcolor[rgb]{0.25,0.50,0.50}{##1}}}
\expandafter\def\csname PY@tok@cs\endcsname{\let\PY@it=\textit\def\PY@tc##1{\textcolor[rgb]{0.25,0.50,0.50}{##1}}}

\def\PYZbs{\char`\\}
\def\PYZus{\char`\_}
\def\PYZob{\char`\{}
\def\PYZcb{\char`\}}
\def\PYZca{\char`\^}
\def\PYZam{\char`\&}
\def\PYZlt{\char`\<}
\def\PYZgt{\char`\>}
\def\PYZsh{\char`\#}
\def\PYZpc{\char`\%}
\def\PYZdl{\char`\$}
\def\PYZhy{\char`\-}
\def\PYZsq{\char`\'}
\def\PYZdq{\char`\"}
\def\PYZti{\char`\~}
% for compatibility with earlier versions
\def\PYZat{@}
\def\PYZlb{[}
\def\PYZrb{]}
\makeatother


    % For linebreaks inside Verbatim environment from package fancyvrb. 
    \makeatletter
        \newbox\Wrappedcontinuationbox 
        \newbox\Wrappedvisiblespacebox 
        \newcommand*\Wrappedvisiblespace {\textcolor{red}{\textvisiblespace}} 
        \newcommand*\Wrappedcontinuationsymbol {\textcolor{red}{\llap{\tiny$\m@th\hookrightarrow$}}} 
        \newcommand*\Wrappedcontinuationindent {3ex } 
        \newcommand*\Wrappedafterbreak {\kern\Wrappedcontinuationindent\copy\Wrappedcontinuationbox} 
        % Take advantage of the already applied Pygments mark-up to insert 
        % potential linebreaks for TeX processing. 
        %        {, <, #, %, $, ' and ": go to next line. 
        %        _, }, ^, &, >, - and ~: stay at end of broken line. 
        % Use of \textquotesingle for straight quote. 
        \newcommand*\Wrappedbreaksatspecials {% 
            \def\PYGZus{\discretionary{\char`\_}{\Wrappedafterbreak}{\char`\_}}% 
            \def\PYGZob{\discretionary{}{\Wrappedafterbreak\char`\{}{\char`\{}}% 
            \def\PYGZcb{\discretionary{\char`\}}{\Wrappedafterbreak}{\char`\}}}% 
            \def\PYGZca{\discretionary{\char`\^}{\Wrappedafterbreak}{\char`\^}}% 
            \def\PYGZam{\discretionary{\char`\&}{\Wrappedafterbreak}{\char`\&}}% 
            \def\PYGZlt{\discretionary{}{\Wrappedafterbreak\char`\<}{\char`\<}}% 
            \def\PYGZgt{\discretionary{\char`\>}{\Wrappedafterbreak}{\char`\>}}% 
            \def\PYGZsh{\discretionary{}{\Wrappedafterbreak\char`\#}{\char`\#}}% 
            \def\PYGZpc{\discretionary{}{\Wrappedafterbreak\char`\%}{\char`\%}}% 
            \def\PYGZdl{\discretionary{}{\Wrappedafterbreak\char`\$}{\char`\$}}% 
            \def\PYGZhy{\discretionary{\char`\-}{\Wrappedafterbreak}{\char`\-}}% 
            \def\PYGZsq{\discretionary{}{\Wrappedafterbreak\textquotesingle}{\textquotesingle}}% 
            \def\PYGZdq{\discretionary{}{\Wrappedafterbreak\char`\"}{\char`\"}}% 
            \def\PYGZti{\discretionary{\char`\~}{\Wrappedafterbreak}{\char`\~}}% 
        } 
        % Some characters . , ; ? ! / are not pygmentized. 
        % This macro makes them "active" and they will insert potential linebreaks 
        \newcommand*\Wrappedbreaksatpunct {% 
            \lccode`\~`\.\lowercase{\def~}{\discretionary{\hbox{\char`\.}}{\Wrappedafterbreak}{\hbox{\char`\.}}}% 
            \lccode`\~`\,\lowercase{\def~}{\discretionary{\hbox{\char`\,}}{\Wrappedafterbreak}{\hbox{\char`\,}}}% 
            \lccode`\~`\;\lowercase{\def~}{\discretionary{\hbox{\char`\;}}{\Wrappedafterbreak}{\hbox{\char`\;}}}% 
            \lccode`\~`\:\lowercase{\def~}{\discretionary{\hbox{\char`\:}}{\Wrappedafterbreak}{\hbox{\char`\:}}}% 
            \lccode`\~`\?\lowercase{\def~}{\discretionary{\hbox{\char`\?}}{\Wrappedafterbreak}{\hbox{\char`\?}}}% 
            \lccode`\~`\!\lowercase{\def~}{\discretionary{\hbox{\char`\!}}{\Wrappedafterbreak}{\hbox{\char`\!}}}% 
            \lccode`\~`\/\lowercase{\def~}{\discretionary{\hbox{\char`\/}}{\Wrappedafterbreak}{\hbox{\char`\/}}}% 
            \catcode`\.\active
            \catcode`\,\active 
            \catcode`\;\active
            \catcode`\:\active
            \catcode`\?\active
            \catcode`\!\active
            \catcode`\/\active 
            \lccode`\~`\~ 	
        }
    \makeatother

    \let\OriginalVerbatim=\Verbatim
    \makeatletter
    \renewcommand{\Verbatim}[1][1]{%
        %\parskip\z@skip
        \sbox\Wrappedcontinuationbox {\Wrappedcontinuationsymbol}%
        \sbox\Wrappedvisiblespacebox {\FV@SetupFont\Wrappedvisiblespace}%
        \def\FancyVerbFormatLine ##1{\hsize\linewidth
            \vtop{\raggedright\hyphenpenalty\z@\exhyphenpenalty\z@
                \doublehyphendemerits\z@\finalhyphendemerits\z@
                \strut ##1\strut}%
        }%
        % If the linebreak is at a space, the latter will be displayed as visible
        % space at end of first line, and a continuation symbol starts next line.
        % Stretch/shrink are however usually zero for typewriter font.
        \def\FV@Space {%
            \nobreak\hskip\z@ plus\fontdimen3\font minus\fontdimen4\font
            \discretionary{\copy\Wrappedvisiblespacebox}{\Wrappedafterbreak}
            {\kern\fontdimen2\font}%
        }%
        
        % Allow breaks at special characters using \PYG... macros.
        \Wrappedbreaksatspecials
        % Breaks at punctuation characters . , ; ? ! and / need catcode=\active 	
        \OriginalVerbatim[#1,codes*=\Wrappedbreaksatpunct]%
    }
    \makeatother

    % Exact colors from NB
    \definecolor{incolor}{HTML}{303F9F}
    \definecolor{outcolor}{HTML}{D84315}
    \definecolor{cellborder}{HTML}{CFCFCF}
    \definecolor{cellbackground}{HTML}{F7F7F7}
    
    % prompt
    \makeatletter
    \newcommand{\boxspacing}{\kern\kvtcb@left@rule\kern\kvtcb@boxsep}
    \makeatother
    \newcommand{\prompt}[4]{
        \ttfamily\llap{{\color{#2}[#3]:\hspace{3pt}#4}}\vspace{-\baselineskip}
    }
    

    
    % Prevent overflowing lines due to hard-to-break entities
    \sloppy 
    % Setup hyperref package
    \hypersetup{
      breaklinks=true,  % so long urls are correctly broken across lines
      colorlinks=true,
      urlcolor=urlcolor,
      linkcolor=linkcolor,
      citecolor=citecolor,
      }
    % Slightly bigger margins than the latex defaults
    
    \geometry{verbose,tmargin=1in,bmargin=1in,lmargin=1in,rmargin=1in}


    

\begin{document}
    
\begin{titlepage}
	\centering
    \vspace*{0.5 cm}
    \includegraphics[scale = 0.50]{logo.png}\\[1.0 cm]	% University Logo
	\text{\large Universidade do Minho}\\[0.1 cm]
	\text{\large Escola de Engenharia}\\[3.0 cm]
	\text{\LARGE Estruturas Criptográficas}\\[0.5 cm]				% Course Code
	\rule{\linewidth}{0.2 mm} \\[0.4 cm]
	{ \huge \bfseries Criptosistemas pós-quânticos PKE/KEM}\\[0.3 cm]
	{ \LARGE Implementação esquemas KEM \textbf{NTRU-Prime} e \textbf{NewHope} em Python/SageMath}
	\rule{\linewidth}{0.2 mm} \\[3.5 cm]
	
	\begin{minipage}{0.4\textwidth}
		\begin{flushleft} \large
			\emph{\textbf{Submitted To:}}\\
			José Valença\\
            Professor Catedrático\\
             Tecnologias da Informação e Segurança\\
			\end{flushleft}
			\end{minipage}~
			\begin{minipage}{0.4\textwidth}
            
			\begin{flushright} \large
			\emph{\textbf{Submitted By :}} \\
			Diogo Araújo, A78485\\
		Diogo Nogueira, A78957\\
            Group 4\\
		\end{flushright}
        
	\end{minipage}\\[2 cm]

	
\end{titlepage}
    
   
    \newpage
	  \hypersetup{linkcolor=black}
\vfill

	\tableofcontents
\vfill

	\newpage




    \section{\texorpdfstring{Implementação do NTRU-Prime com o
\emph{SageMath}}{Implementação do NTRU-Prime com o SageMath}}\label{implementauxe7uxe3o-do-ntru-prime-com-o-sagemath}
\vspace{5mm}
\subsection{Descrição do
Exercício}\label{descriuxe7uxe3o-do-exercuxedcio}
\vspace{2mm}
A ideia do exercício passa por construir toda uma classe Python/SageMath
que implemente o esquema KEM \textbf{NTRU-Prime}, algoritmo candidato ao
\textbf{NIST's \emph{Post-Quantum Cryptography Standardization
Project}}.

Com esse fim em mente, o grupo recorreu à documentação de candidatura do
algoritmo fornecida pelo documento, mais especificamente ao documento de
envio principal de seu nome \textbf{"NTRU Prime"}, submitido no dia
30/11/2017.
\vspace{3mm}
\subsection{Descrição da
Implementação}\label{descriuxe7uxe3o-da-implementauxe7uxe3o}
\vspace{2mm}
Com o \emph{Primary Submission Document} em mão e com uma análise atenta
de todo o processo algorítmico, podem-se definir o conjunto de
definições que a classe Python terá e que permitirão no final fazer um
pequeno teste em termos de resultados, para as versões PKE-IND-CCA e
KEM-IND-CPA. De forma a faciliar a compreensão/execução de todas as
etapas pensadas para o algoritmo, criou-se a divisão que consta no
próprio documento em análise.
\vspace{3mm}

\textbf{Com a pesquisa necessária e com a ideia do funcionamento do
algoritmo em mente, estabelecem-se as seguintes implementações:}
\vspace{2mm}

\begin{enumerate}
\def\labelenumi{\arabic{enumi}.}
\item
  \textbf{\emph{Parameter Space} - Criação/Geração dos Parâmetros e de
  todos os Anéis de Polinómios necessários para o restante do programa:}

  \begin{itemize}
  \tightlist
  \item
    Criação do parâmetro \(q\) - que corresponde a um número primo
  \item
    Criação do parâmetro \(p\) - que corresponde também a um número
    primo
  \item
    O parâmetro \(w\) é criado e usado em toda a parte da \emph{Key
    Generation}, \emph{Encapsulation} e \emph{Desencapsulation}
  \item
    Verificar se \(x^{p}-x-1\) é irredutível no Anel de Polinómios
    \((\mathbb{Z}/q)[x]\)
  \item
    Abreviar os 2 Anéis de Polinónimo \(\mathbb{Z}[x]/(x^{p}-x-1)\) e
    \((\mathbb{Z}/3)[x]/(x^{p}-x-1)\) e o Campo/Grupo Finito
    \((\mathbb{Z}/q)[x]/(x^{p}-x-1)\) respetivamente como \({R}\),
    \({R}/3\) e \({R}/q\)
  \end{itemize}

  \textbf{Note-se que o \emph{field} criado é usado para verificar a
  irredutibilidade de \(x^{p}-x-1\), justificando-se assim a ordem a
  nível do código.}
\vspace{3mm}

\item
  \textbf{\emph{Key Generation} - Geração das Chaves (\emph{Secret} e
  \emph{Public Key}):}

  \begin{itemize}
  \tightlist
  \item
    Criação de um elemento pequeno \(g\) uniforme e aleatório tal que
    \(g\in{R}\). Repetir o processo de modo a verificar que \(g\) é
    invertível em \({R}/3\)
  \item
    Criação de um elemento pequeno \(f\) uniforme e aleatório tal que
    \(f\in{R}\). Este elemento \(f\) deve ser de peso \(w\), diferente
    de 0 e por isso, invertível em \({R}/q\)
  \item
    O Segredo guardado será o \textbf{par ordenado} \((f, 1/g)\) em que
    \(f\) em \({R}\) e \(1/g\) em \({R}/3\)
  \item
    A Chave Pública corresponderá ao cálculo \(h = g/(3f)\) em \({R}/q\)
  \end{itemize}
\vspace{3mm}

\item
  \textbf{\emph{Encapsulation} - Processo de Encapsulamento:}

  \begin{itemize}
  \tightlist
  \item
    Obtenção do \(h\in{R}/q\). Para este teste algorítmico corresponderá
    à \emph{Public Key} dada como parâmetro da função criada para o
    efeito
  \item
    Criação de um elemento pequeno \(e\) uniforme e aleatório tal que
    \(r\in{R}\). Este elemento \(r\) deve ser de peso \(w\)
  \item
    Calcular \(hr\in{R}/q\)
  \item
    Arredondar cada coeficiente de \(hr\) visto como um inteiro entre
    \(-(q-1)/2\) e \((q-1)/2\), para o múltiplo de 3 mais próximo,
    obtendo-se assim \(C\in{R}\)
  \end{itemize}

  \textbf{A parte da Hash é feita na secção de teste criada para esta
  classe Python, de modo a se conseguir testar as versões PKE-IND-CCA e
  KEM-IND-CPA.}
\vspace{3mm}

\item
  \textbf{\emph{Decapsulation} - Processo de Desencapsulamento:}

  \begin{itemize}
  \tightlist
  \item
    Obtenção do valor de \(C\) que é passado como argumento
  \item
    Multiplicar o valor de \(C\) por \(3f\) em \({R}/q\)
  \item
    Arredondar cada coeficiente de \(3fC\) visto como um inteiro entre
    \(-(q-1)/2\) e \((q-1)/2\), reduzindo ao módulo 3, obtendo-se um
    polinómio em \({R}/3\)
  \item
    Multiplicar esse valor por \(1/g\in{R}/3\)
  \item
    Calcular \(r'\)
  \end{itemize}
\end{enumerate}
\vspace{3mm}

\subsection{Resolução do
Exercício}\label{resoluuxe7uxe3o-do-exercuxedcio}
\vspace{2mm}

    \subsubsection{\texorpdfstring{1. \emph{Parameter
Space}}{1. Parameter Space}}\label{parameter-space}

Geração dos Parâmetros e dos Anéis de Polinómios necessários para o
restante do programa. Dado a necessidade de usar estas variáveis ao
longo do algoritmo, criou-se uma célula à parte para permitir a
globalidade do programa de forma correta.

    \begin{Verbatim}[commandchars=\\\{\}]
{\color{incolor}In [{\color{incolor}1}]:} \PY{c+c1}{\PYZsh{} Parâmetro primo q}
        \PY{n}{q} \PY{o}{=} \PY{l+m+mi}{24}\PY{o}{*}\PY{l+m+mi}{64}
        
        \PY{c+c1}{\PYZsh{} Verifica se o valor dado ao q é primo}
        \PY{c+c1}{\PYZsh{} Cria\PYZhy{}se um ciclo while para fazer essa verificação}
        \PY{k}{while} \PY{n+nb+bp}{True}\PY{p}{:}
            
            \PY{k}{if} \PY{p}{(}\PY{l+m+mi}{1} \PY{o}{+} \PY{n}{q}\PY{p}{)}\PY{o}{.}\PY{n}{is\PYZus{}prime}\PY{p}{(}\PY{p}{)}\PY{p}{:}
                \PY{k}{break}
            \PY{k}{else}\PY{p}{:}
                \PY{n}{q} \PY{o}{=} \PY{n}{q} \PY{o}{+} \PY{l+m+mi}{3}
        \PY{n}{q} \PY{o}{=} \PY{n}{q} \PY{o}{+} \PY{l+m+mi}{1}
        
        \PY{c+c1}{\PYZsh{} Anéis de Polinómios e Campo/Grupo Finito}
        \PY{n}{Zx}\PY{o}{.}\PY{o}{\PYZlt{}}\PY{n}{x}\PY{o}{\PYZgt{}} \PY{o}{=} \PY{n}{ZZ}\PY{p}{[}\PY{p}{]}                   \PY{c+c1}{\PYZsh{} Anel de Inteiros}
        \PY{n}{Z3}\PY{o}{.}\PY{o}{\PYZlt{}}\PY{n}{y}\PY{o}{\PYZgt{}} \PY{o}{=} \PY{n}{PolynomialRing}\PY{p}{(}\PY{n}{GF}\PY{p}{(}\PY{l+m+mi}{3}\PY{p}{)}\PY{p}{)}  \PY{c+c1}{\PYZsh{} Anel do Grupo Finito do módulo 3}
        \PY{n}{Gq}\PY{o}{.}\PY{o}{\PYZlt{}}\PY{n}{z}\PY{o}{\PYZgt{}} \PY{o}{=} \PY{n}{GF}\PY{p}{(}\PY{n}{q}\PY{p}{)}\PY{p}{[}\PY{p}{]}                \PY{c+c1}{\PYZsh{} Grupo Finito do módulo q}
        
        \PY{c+c1}{\PYZsh{} Parâmetro primo p}
        \PY{c+c1}{\PYZsh{} Dado que se declarou o primo q anteriormente pode\PYZhy{}se fazer logo uso da função next\PYZus{}prime}
        \PY{n}{p} \PY{o}{=} \PY{n}{next\PYZus{}prime}\PY{p}{(}\PY{l+m+mi}{2}\PY{o}{*}\PY{l+m+mi}{64}\PY{p}{)}
        
        \PY{c+c1}{\PYZsh{} Verifica se x\PYZca{}p \PYZhy{} x \PYZhy{} 1 é irredutível no Grupo Finito Gq}
        \PY{c+c1}{\PYZsh{} Cria\PYZhy{}se um ciclo while para fazer essa verificação}
        \PY{k}{while} \PY{n+nb+bp}{True}\PY{p}{:}
            
            \PY{k}{if}  \PY{n}{Gq}\PY{p}{(}\PY{n}{x}\PY{o}{\PYZca{}}\PY{n}{p}\PY{o}{\PYZhy{}}\PY{n}{x}\PY{o}{\PYZhy{}}\PY{l+m+mi}{1}\PY{p}{)}\PY{o}{.}\PY{n}{is\PYZus{}irreducible}\PY{p}{(}\PY{p}{)}\PY{p}{:}
                \PY{k}{break}
            \PY{k}{else}\PY{p}{:}
                \PY{n}{p} \PY{o}{=} \PY{n}{next\PYZus{}prime}\PY{p}{(}\PY{n}{p}\PY{o}{+}\PY{l+m+mi}{1}\PY{p}{)}
        
        \PY{c+c1}{\PYZsh{} Resolver/Abreviar os Anéis e o Campo/Grupo como R, R/3 e R/q}
        \PY{n}{ZxR}\PY{o}{.}\PY{o}{\PYZlt{}}\PY{n}{x}\PY{o}{\PYZgt{}} \PY{o}{=} \PY{n}{Zx}\PY{o}{.}\PY{n}{quotient}\PY{p}{(}\PY{n}{x}\PY{o}{\PYZca{}}\PY{n}{p}\PY{o}{\PYZhy{}}\PY{n}{x}\PY{o}{\PYZhy{}}\PY{l+m+mi}{1}\PY{p}{)}
        \PY{n}{ZR3}\PY{o}{.}\PY{o}{\PYZlt{}}\PY{n}{y}\PY{o}{\PYZgt{}} \PY{o}{=} \PY{n}{Z3}\PY{o}{.}\PY{n}{quotient}\PY{p}{(}\PY{n}{y}\PY{o}{\PYZca{}}\PY{n}{p}\PY{o}{\PYZhy{}}\PY{n}{y}\PY{o}{\PYZhy{}}\PY{l+m+mi}{1}\PY{p}{)}
        \PY{n}{GRq}\PY{o}{.}\PY{o}{\PYZlt{}}\PY{n}{z}\PY{o}{\PYZgt{}} \PY{o}{=} \PY{n}{Gq}\PY{o}{.}\PY{n}{quotient}\PY{p}{(}\PY{n}{z}\PY{o}{\PYZca{}}\PY{n}{p}\PY{o}{\PYZhy{}}\PY{n}{z}\PY{o}{\PYZhy{}}\PY{l+m+mi}{1}\PY{p}{)}
\end{Verbatim}
\vspace{2mm}

    \subsubsection{\texorpdfstring{2, 3 e 4. \emph{Key Generation},
\emph{Encapsulation} e
\emph{Decapsulation}}{2, 3 e 4. Key Generation, Encapsulation e Decapsulation}}\label{e-4.-key-generation-encapsulation-e-decapsulation}

    \begin{Verbatim}[commandchars=\\\{\}]
{\color{incolor}In [{\color{incolor}6}]:} \PY{c+c1}{\PYZsh{} Imports Necessários}
        \PY{k+kn}{import} \PY{n+nn}{hashlib}
        \PY{k+kn}{from} \PY{n+nn}{random} \PY{k+kn}{import} \PY{n}{choice}\PY{p}{,} \PY{n}{randint}
        
        \PY{c+c1}{\PYZsh{} Função auxiliar que gera um Pequeno Elemento Aleatório}
        \PY{k}{def} \PY{n+nf}{smallPoly}\PY{p}{(}\PY{n}{p}\PY{p}{,} \PY{n}{t} \PY{o}{=} \PY{n+nb+bp}{None}\PY{p}{)}\PY{p}{:}
        
            \PY{k}{if} \PY{o+ow}{not} \PY{n}{t}\PY{p}{:}
                \PY{k}{return} \PY{n}{Zx}\PY{p}{(}\PY{p}{[}\PY{n}{choice}\PY{p}{(}\PY{p}{[}\PY{o}{\PYZhy{}}\PY{l+m+mi}{1}\PY{p}{,}\PY{l+m+mi}{0}\PY{p}{,}\PY{l+m+mi}{1}\PY{p}{]}\PY{p}{)} \PY{k}{for} \PY{n}{k} \PY{o+ow}{in} \PY{n+nb}{range}\PY{p}{(}\PY{n}{p}\PY{p}{)}\PY{p}{]}\PY{p}{)}
        
            \PY{n}{u} \PY{o}{=} \PY{n}{floor}\PY{p}{(}\PY{l+m+mi}{2}\PY{o}{*}\PY{p}{(}\PY{n}{p}\PY{o}{\PYZhy{}}\PY{l+m+mi}{1}\PY{p}{)}\PY{o}{/}\PY{o}{/}\PY{n}{t}\PY{p}{)} 
            \PY{n}{k} \PY{o}{=} \PY{n}{randint}\PY{p}{(}\PY{l+m+mi}{0}\PY{p}{,} \PY{n}{u}\PY{p}{)} 
            \PY{n}{l} \PY{o}{=} \PY{p}{[}\PY{l+m+mi}{0}\PY{p}{]}\PY{o}{*}\PY{n}{p}
        
            \PY{k}{while} \PY{n}{k} \PY{o}{\PYZlt{}} \PY{n}{p}\PY{p}{:}
        
                \PY{n}{l}\PY{p}{[}\PY{n}{k}\PY{p}{]} \PY{o}{=} \PY{n}{choice}\PY{p}{(}\PY{p}{[}\PY{o}{\PYZhy{}}\PY{l+m+mi}{1}\PY{p}{,} \PY{l+m+mi}{1}\PY{p}{]}\PY{p}{)}
                \PY{n}{k} \PY{o}{+}\PY{o}{=} \PY{n}{randint}\PY{p}{(}\PY{l+m+mi}{1}\PY{p}{,} \PY{n}{u}\PY{p}{)}
        
            \PY{k}{return} \PY{n}{Zx}\PY{p}{(}\PY{n}{l}\PY{p}{)}
        
        \PY{c+c1}{\PYZsh{} Função auxiliar que trata de arredondar cada coeficiente visto como um inteiro entre \PYZhy{}(q\PYZhy{}1)/2 e (q\PYZhy{}1)/2}
        \PY{c+c1}{\PYZsh{} Função adaptada para funcionar tanto para o Encapsulamento como Desencapsulamento}
        \PY{k}{def} \PY{n+nf}{myRound}\PY{p}{(}\PY{n}{round3OrNot}\PY{p}{,} \PY{n}{hr} \PY{o}{=} \PY{n+nb+bp}{None}\PY{p}{,} \PY{n}{w} \PY{o}{=} \PY{n+nb+bp}{None}\PY{p}{,} \PY{n}{q} \PY{o}{=} \PY{n}{q}\PY{p}{)}\PY{p}{:}          
        
            \PY{c+c1}{\PYZsh{} Caso seja 0 significa que estamos a tratar do arredondamento pedido pelo Desencapsulamento}
            \PY{k}{if} \PY{n}{round3OrNot} \PY{o}{==} \PY{l+m+mi}{0}\PY{p}{:}
                \PY{n}{r} \PY{o}{=} \PY{n}{q}\PY{o}{/}\PY{o}{/}\PY{l+m+mi}{2}
                
                \PY{k}{return} \PY{n}{Zx}\PY{p}{(}\PY{n+nb}{map}\PY{p}{(}\PY{k}{lambda} \PY{n}{x}\PY{p}{:} \PY{n}{lift}\PY{p}{(}\PY{n}{x} \PY{o}{+} \PY{n}{r}\PY{p}{)} \PY{o}{\PYZhy{}} \PY{n}{r}\PY{p}{,} \PY{n}{w}\PY{o}{.}\PY{n}{list}\PY{p}{(}\PY{p}{)}\PY{p}{)}\PY{p}{)}
            
            \PY{c+c1}{\PYZsh{} Caso contrário, pelo Encapsulamento}
            \PY{c+c1}{\PYZsh{} Daí o arredondamento para múltiplo de 3 mais próximo}
            \PY{k}{elif} \PY{n}{round3OrNot} \PY{o}{==} \PY{l+m+mi}{1}\PY{p}{:}
                
                \PY{c+c1}{\PYZsh{} Aredondar para multpilo de 3 mais próximo}
                \PY{k}{def} \PY{n+nf}{mul3}\PY{p}{(}\PY{n}{x}\PY{p}{)}\PY{p}{:}
                    \PY{k}{return} \PY{p}{(}\PY{p}{(}\PY{n}{x}\PY{o}{/}\PY{l+m+mi}{3}\PY{p}{)}\PY{o}{.}\PY{n}{round}\PY{p}{(}\PY{p}{)}\PY{p}{)}\PY{o}{*}\PY{l+m+mi}{3}
        
                \PY{n}{r} \PY{o}{=} \PY{n}{q}\PY{o}{/}\PY{o}{/}\PY{l+m+mi}{2}
        
                \PY{k}{return}  \PY{n}{Zx}\PY{p}{(}\PY{n+nb}{map}\PY{p}{(}\PY{k}{lambda} \PY{n}{x}\PY{p}{:} \PY{n}{mul3}\PY{p}{(}\PY{n}{lift}\PY{p}{(}\PY{n}{x}\PY{o}{+}\PY{n}{r}\PY{p}{)} \PY{o}{\PYZhy{}} \PY{n}{r}\PY{p}{)}\PY{p}{,} \PY{n}{hr}\PY{o}{.}\PY{n}{list}\PY{p}{(}\PY{p}{)}\PY{p}{)}\PY{p}{)}
        
        \PY{c+c1}{\PYZsh{} Classe NTRU\PYZhy{}Prime que trata da Geração das Chaves, Encapsulamento e Desencapsulamento}
        \PY{k}{class} \PY{n+nc}{NTRU\PYZus{}Prime}\PY{p}{:}   
        
            \PY{c+c1}{\PYZsh{} Função de Hash}
            \PY{k}{def} \PY{n+nf}{Hash}\PY{p}{(}\PY{n+nb+bp}{self}\PY{p}{,} \PY{n}{w}\PY{p}{)}\PY{p}{:}
                \PY{n}{ww} \PY{o}{=} \PY{n+nb}{reduce}\PY{p}{(}\PY{k}{lambda} \PY{n}{x}\PY{p}{,}\PY{n}{y}\PY{p}{:} \PY{n}{x} \PY{o}{+} \PY{n}{y}\PY{o}{.}\PY{n}{binary}\PY{p}{(}\PY{p}{)}\PY{p}{,} \PY{n}{w}\PY{o}{.}\PY{n}{list}\PY{p}{(}\PY{p}{)}\PY{p}{,} \PY{l+s+s2}{\PYZdq{}}\PY{l+s+s2}{\PYZdq{}}\PY{p}{)}
        
                \PY{k}{return} \PY{n}{hashlib}\PY{o}{.}\PY{n}{sha256}\PY{p}{(}\PY{n}{ww}\PY{p}{)}\PY{o}{.}\PY{n}{hexdigest}\PY{p}{(}\PY{p}{)}
            
            \PY{c+c1}{\PYZsh{} Geração das Chaves (Segredo e Public Key)}
            \PY{k}{def} \PY{n+nf}{keyGeneration}\PY{p}{(}\PY{n+nb+bp}{self}\PY{p}{)}\PY{p}{:}
        
                \PY{c+c1}{\PYZsh{} Pequeno Elemento aleatório g pertecente a R}
                \PY{c+c1}{\PYZsh{} Verificar que g seja invertível em R/3}
                \PY{n}{g} \PY{o}{=} \PY{n}{smallPoly}\PY{p}{(}\PY{n}{p}\PY{p}{)}
                
                \PY{c+c1}{\PYZsh{} Daí se recorrer ao Anel de Polinómios (Z/3)[x]}
                \PY{k}{while} \PY{o+ow}{not} \PY{n}{ZR3}\PY{p}{(}\PY{n}{g}\PY{p}{)}\PY{o}{.}\PY{n}{is\PYZus{}unit}\PY{p}{(}\PY{p}{)}\PY{p}{:}   
                    \PY{n}{g} \PY{o}{=} \PY{n}{smallPoly}\PY{p}{(}\PY{n}{p}\PY{p}{)}
        
                \PY{c+c1}{\PYZsh{} Pequeno Elemento aleatório f pertencente a R de peso w}
                \PY{c+c1}{\PYZsh{} Verificar que f é diferente de 0 e invertível em R/q}
                \PY{n}{w} \PY{o}{=} \PY{n}{smallPoly}\PY{p}{(}\PY{n}{p}\PY{p}{,} \PY{l+m+mi}{64}\PY{p}{)}
                \PY{n}{f} \PY{o}{=} \PY{n}{GRq}\PY{p}{(}\PY{n}{w}\PY{p}{)}
                
                \PY{c+c1}{\PYZsh{} Calcular 1/g pertencente a R/3}
                \PY{n}{gInv} \PY{o}{=} \PY{n}{ZR3}\PY{p}{(}\PY{n}{g}\PY{p}{)}\PY{o}{\PYZca{}}\PY{p}{(}\PY{o}{\PYZhy{}}\PY{l+m+mi}{1}\PY{p}{)}
        
                \PY{c+c1}{\PYZsh{} Chave Privada (Secrets)}
                \PY{c+c1}{\PYZsh{} Par Ordenado (f, 1/g)}
                \PY{n}{secret} \PY{o}{=} \PY{p}{(}\PY{n}{f}\PY{p}{,} \PY{n}{gInv}\PY{p}{)}
        
                \PY{c+c1}{\PYZsh{} Chave Pública em Gqr (h)}
                \PY{n}{publicKey} \PY{o}{=} \PY{n}{GRq}\PY{p}{(}\PY{n}{g}\PY{p}{)}\PY{o}{/}\PY{n}{GRq}\PY{p}{(}\PY{l+m+mi}{3}\PY{o}{*}\PY{n}{f}\PY{p}{)}
        
                \PY{k}{return} \PY{p}{(}\PY{n}{secret}\PY{p}{,} \PY{n}{publicKey}\PY{p}{)}
        
            \PY{c+c1}{\PYZsh{} Encapsulamento}
            \PY{k}{def} \PY{n+nf}{encapsulate}\PY{p}{(}\PY{n+nb+bp}{self}\PY{p}{,} \PY{n}{publicKey}\PY{p}{)}\PY{p}{:}
        
                \PY{c+c1}{\PYZsh{} Public Key h}
                \PY{c+c1}{\PYZsh{} Neste caso é o argumento passado como pk}
                \PY{n}{h} \PY{o}{=} \PY{n}{publicKey}
                
                \PY{c+c1}{\PYZsh{} Pequeno Elemento aleatório f pertencente a R de peso w}
                \PY{n}{w} \PY{o}{=} \PY{n}{smallPoly}\PY{p}{(}\PY{n}{p}\PY{p}{,} \PY{l+m+mi}{64}\PY{p}{)}
                \PY{n}{r} \PY{o}{=} \PY{n}{GRq}\PY{p}{(}\PY{n}{w}\PY{p}{)}
                
                \PY{c+c1}{\PYZsh{} Calcular hr}
                \PY{c+c1}{\PYZsh{} Arredondar cada coeficiente de hr}
                \PY{n}{C} \PY{o}{=} \PY{n}{myRound}\PY{p}{(}\PY{l+m+mi}{1}\PY{p}{,} \PY{n}{h}\PY{o}{*}\PY{n}{r}\PY{p}{)}
        
                \PY{k}{return} \PY{p}{(}\PY{n}{w}\PY{p}{,} \PY{n}{C}\PY{p}{)}
            
            \PY{c+c1}{\PYZsh{} Desencapsulamento}
            \PY{k}{def} \PY{n+nf}{decapsulate}\PY{p}{(}\PY{n+nb+bp}{self}\PY{p}{,} \PY{n}{secret}\PY{p}{,} \PY{n}{C}\PY{p}{)}\PY{p}{:}
                
                \PY{c+c1}{\PYZsh{} Obter o valor de f e 1/g}
                \PY{p}{(}\PY{n}{f}\PY{p}{,} \PY{n}{gInv}\PY{p}{)} \PY{o}{=} \PY{n}{secret}
                
                \PY{c+c1}{\PYZsh{} Multiplicar C por 3f em R/q}
                \PY{c+c1}{\PYZsh{} Arredondar cada coeficiente de 3fc}
                \PY{c+c1}{\PYZsh{} Multiplicar por 1/g (gInv)}
                \PY{n}{e} \PY{o}{=} \PY{n}{gInv} \PY{o}{*} \PY{n}{ZR3}\PY{p}{(}\PY{n}{myRound}\PY{p}{(}\PY{l+m+mi}{0}\PY{p}{,} \PY{n+nb+bp}{None}\PY{p}{,} \PY{n}{GRq}\PY{p}{(}\PY{l+m+mi}{3}\PY{o}{*}\PY{n}{f}\PY{p}{)} \PY{o}{*} \PY{n}{GRq}\PY{p}{(}\PY{n}{C}\PY{p}{)}\PY{p}{)}\PY{p}{)}
                
                \PY{c+c1}{\PYZsh{} Calcular r\PYZsq{}}
                \PY{n}{rlinha} \PY{o}{=} \PY{n}{myRound}\PY{p}{(}\PY{l+m+mi}{0}\PY{p}{,} \PY{n+nb+bp}{None}\PY{p}{,} \PY{n}{e}\PY{p}{,} \PY{l+m+mi}{3}\PY{p}{)}
                
                \PY{k}{return} \PY{n}{rlinha}
\end{Verbatim}
\vspace{3mm}

    \subsection{Teste da Classe}\label{teste-da-classe}
\vspace{2mm}

    \textbf{Teste e Verificação para o modo PKE-IND-CCA}

    \begin{Verbatim}[commandchars=\\\{\}]
{\color{incolor}In [{\color{incolor}22}]:} \PY{c+c1}{\PYZsh{} Criação da Instância}
         \PY{n}{ntruPrime} \PY{o}{=} \PY{n}{NTRU\PYZus{}Prime}\PY{p}{(}\PY{p}{)}
         
         \PY{c+c1}{\PYZsh{} Geração das Chaves (Segredo e Chave Pública)}
         \PY{p}{(}\PY{n}{secret}\PY{p}{,} \PY{n}{publicKey}\PY{p}{)} \PY{o}{=} \PY{n}{ntruPrime}\PY{o}{.}\PY{n}{keyGeneration}\PY{p}{(}\PY{p}{)}
         
         \PY{c+c1}{\PYZsh{} Cifrar}
         \PY{p}{(}\PY{n}{key}\PY{p}{,}\PY{n}{C}\PY{p}{)} \PY{o}{=} \PY{n}{ntruPrime}\PY{o}{.}\PY{n}{encapsulate}\PY{p}{(}\PY{n}{publicKey}\PY{p}{)} 
         
         \PY{c+c1}{\PYZsh{} Decifrar e Verificar}
         \PY{k}{if} \PY{n}{key} \PY{o}{==} \PY{n}{ntruPrime}\PY{o}{.}\PY{n}{decapsulate}\PY{p}{(}\PY{n}{secret}\PY{p}{,} \PY{n}{C}\PY{p}{)}\PY{p}{:}
             \PY{k}{print} \PY{l+s+s2}{\PYZdq{}}\PY{l+s+s2}{As chaves correspondem.}\PY{l+s+s2}{\PYZdq{}}
             \PY{k}{print}\PY{p}{(}\PY{n}{key}\PY{p}{)}
         \PY{k}{else}\PY{p}{:}
             \PY{k}{print} \PY{l+s+s2}{\PYZdq{}}\PY{l+s+s2}{Erro na correspondência das chaves.}\PY{l+s+s2}{\PYZdq{}}
\end{Verbatim}


    \begin{Verbatim}[commandchars=\\\{\}]
As chaves correspondem.
-x\^{}246 + x\^{}243 + x\^{}242 + x\^{}240 - x\^{}239 - x\^{}235 - x\^{}229 - x\^{}224 + x\^{}217 + x\^{}213 - x\^{}206 + x\^{}203 + x\^{}198 - x\^{}196 - x\^{}195 - x\^{}194 + x\^{}189 + x\^{}186 + x\^{}181 - x\^{}180 + x\^{}175 + x\^{}170 - x\^{}166 - x\^{}163 - x\^{}159 + x\^{}156 - x\^{}151 + x\^{}148 + x\^{}143 + x\^{}140 + x\^{}137 + x\^{}133 - x\^{}129 + x\^{}123 - x\^{}117 - x\^{}116 - x\^{}109 + x\^{}108 + x\^{}105 - x\^{}104 + x\^{}97 + x\^{}90 + x\^{}84 + x\^{}83 + x\^{}78 - x\^{}76 - x\^{}75 - x\^{}73 + x\^{}66 - x\^{}60 + x\^{}54 - x\^{}52 - x\^{}46 + x\^{}39 + x\^{}38 - x\^{}31 + x\^{}30 + x\^{}25 - x\^{}21 - x\^{}19 + x\^{}18 - x\^{}17 + x\^{}14 + x\^{}8 - x\^{}6 + x\^{}5 + x\^{}2 + x

    \end{Verbatim}
\newpage
    \textbf{Teste e Verficação da versão KEM-IND-CPA}
\vspace{2mm}

KEM é a randomização duma chave do tamanho correto do algoritmo
utilizado (neste caso o NTRU-Prime) e depois a chave simétrica a
utilizar futuramente, na verdade está encapsulada utilizando um KDF (ex:
SHA-256). Assim esse KDF irá derivar a chave "grande" do PKE para uma
chave simétrica com o tamanho correto (SHA-256 dá valores de 256-bits,
que podem ser utilizados num algoritmo simétrico como AES-256)

    \begin{Verbatim}[commandchars=\\\{\}]
{\color{incolor}In [{\color{incolor}23}]:} \PY{c+c1}{\PYZsh{} Criação da Instância}
         \PY{n}{ntruPrime} \PY{o}{=} \PY{n}{NTRU\PYZus{}Prime}\PY{p}{(}\PY{p}{)}
         
         \PY{c+c1}{\PYZsh{} Geração das Chaves (Segredo e Chave Pública)}
         \PY{p}{(}\PY{n}{secret}\PY{p}{,} \PY{n}{publicKey}\PY{p}{)} \PY{o}{=} \PY{n}{ntruPrime}\PY{o}{.}\PY{n}{keyGeneration}\PY{p}{(}\PY{p}{)}
         
         \PY{c+c1}{\PYZsh{} 1. Bob}
         
         \PY{c+c1}{\PYZsh{} Cifrar}
         \PY{p}{(}\PY{n}{keyBob}\PY{p}{,} \PY{n}{C}\PY{p}{)} \PY{o}{=} \PY{n}{ntruPrime}\PY{o}{.}\PY{n}{encapsulate}\PY{p}{(}\PY{n}{publicKey}\PY{p}{)}
         
         \PY{c+c1}{\PYZsh{} Passar a Chave de tamanho grande do PKE NTRU\PYZhy{}Prime para 256 bits usando SHA\PYZhy{}256}
         \PY{n}{chaveSimetricaBob} \PY{o}{=} \PY{n}{ntruPrime}\PY{o}{.}\PY{n}{Hash}\PY{p}{(}\PY{n}{keyBob}\PY{p}{)}
         
         \PY{c+c1}{\PYZsh{} 2. Alice}
         
         \PY{c+c1}{\PYZsh{} Cifrar}
         \PY{n}{keyAlice} \PY{o}{=} \PY{n}{ntruPrime}\PY{o}{.}\PY{n}{decapsulate}\PY{p}{(}\PY{n}{secret}\PY{p}{,} \PY{n}{C}\PY{p}{)}
         
         \PY{c+c1}{\PYZsh{} Se as chaves PKE coincidirem (Algoritmo funciona corretamente)}
         \PY{k}{if} \PY{n}{keyBob} \PY{o}{==} \PY{n}{keyAlice}\PY{p}{:}
             \PY{n}{chaveSimetricaAlice} \PY{o}{=} \PY{n}{ntruPrime}\PY{o}{.}\PY{n}{Hash}\PY{p}{(}\PY{n}{keyAlice}\PY{p}{)}
         \PY{k}{else}\PY{p}{:}
             \PY{k}{print} \PY{l+s+s2}{\PYZdq{}}\PY{l+s+s2}{O Algoritmo NTRU não decifrou corretamente.}\PY{l+s+s2}{\PYZdq{}}
         
         \PY{c+c1}{\PYZsh{} Verificar a validade de ambas as Chaves Simétricas}
         \PY{k}{if} \PY{n}{chaveSimetricaBob} \PY{o}{==} \PY{n}{chaveSimetricaAlice}\PY{p}{:}
             \PY{k}{print} \PY{l+s+s2}{\PYZdq{}}\PY{l+s+s2}{As chaves correspondem.}\PY{l+s+s2}{\PYZdq{}}
             \PY{k}{print}\PY{p}{(}\PY{n}{chaveSimetricaBob}\PY{p}{)}
         \PY{k}{else}\PY{p}{:}
             \PY{k}{print} \PY{l+s+s2}{\PYZdq{}}\PY{l+s+s2}{Erro na correspondência das chaves.}\PY{l+s+s2}{\PYZdq{}}
\end{Verbatim}


    \begin{Verbatim}[commandchars=\\\{\}]
As chaves correspondem.
74c63fbed90859b8d5198d8ec717686a655ed1e75ad84bce791f56d76ff03497

    \end{Verbatim}

    \subsection{Observações Finais}\label{observauxe7uxf5es-finais}
\vspace{2mm}

\begin{itemize}
\tightlist
\item
  O algoritmo \textbf{NTRU} ajudou a ter uma noção muito basilar acerca
  do funcionamento do \textbf{NTRU-Prime};
\item
  Através do documento de envio principal do \textbf{NTRU-Prime}
  consegui-se compreender bem todos os passos e com isso criar as várias
  etapas que formam o algoritmo, permitindo testar toda a validade do
  programa Python/Sagemath;
\item
  A grande dificuldade continua a ser, tal como nos outros Trabalhos
  Práticos, toda a ideia de trabalhar com o Sagemath e compreender todas
  as suas funcionalidades necessárias para os vários algoritmos
  criptográficos.
\end{itemize}
\vspace{2mm}

    \subsection{Referências}\label{referuxeancias}
\vspace{2mm}

\begin{itemize}
\tightlist
\item
  ENS DE LYON, NTRU Prime: Intro \url{https://ntruprime.cr.yp.to} (Acedido a
  10 maio 2020)
\item
  Wikipedia, NTRU \url{https://en.wikipedia.org/wiki/NTRU} (Acedido a 11 maio
  2020)
\end{itemize}
\vspace{5mm}

\newpage
    \section{\texorpdfstring{Implementação do NewHope com o
\emph{SageMath}}{Implementação do NewHope com o SageMath}}\label{implementauxe7uxe3o-do-newhope-com-o-sagemath}
\vspace{3mm}

\subsection{Descrição do
Exercício}\label{descriuxe7uxe3o-do-exercuxedcio}
\vspace{2mm}
Construir uma classe Python/SageMath que implemente o esquema KEM
\textbf{NewHope} a concurso do NIST-PQC.
\vspace{2mm}
\subsection{Descrição da
Implementação}\label{descriuxe7uxe3o-da-implementauxe7uxe3o}
\vspace{2mm}
Para a implementação houve a criação de um campo ciclotómico, através
dum anel polinomial dum grupo finito de \(1024\), logo ficando
\(R/[X^1024+1]\).
\vspace{2mm}

\begin{enumerate}
\def\labelenumi{\arabic{enumi}.}
\item
  \textbf{Para a inicialização do algoritmo NewHope fez-se:}

  \begin{itemize}
  \tightlist
  \item
    Uma matriz identidade de sub-dimensão 4
  \item
    Uma rede diagonal de inteiros a partir dessa matriz
  \item
    Criar um meio vetor e modificar a última linha da matriz com ele
  \item
    Criação do poliedro principal com as estruturas matemáticas
    anteriores
  \end{itemize}
\vspace{2mm}

\item
  \textbf{Para a geração de noise/erro:}

  \begin{itemize}
  \tightlist
  \item
    Buscamos um \emph{sample} polinomial da Distribuição Gaussiana
  \end{itemize}
\vspace{2mm}

\item
  \textbf{Para a geração do sinal:}

  \begin{itemize}
  \tightlist
  \item
    Vamos fazer a divisão dos elementos pelo módulo tendo os
    coeficientes
  \item
    Caso o vetor \(V\) esteja no poliedro principal, utiliza-se. Senão
    vai-se à rede diagonal e usa-se o vetor mais próximo de \(V\).
  \item
    Sendo assim a distância, o sinal.
  \end{itemize}
\vspace{2mm}

\item
  \textbf{Para a reconciliação é um processo análogo:}

  \begin{itemize}
  \tightlist
  \item
    Vamos fazer a divisão dos elementos pelo módulo tendo os
    coeficientes
  \item
    Adicionamos \(1\) caso a coordenada esteja no centro do poliedro,
    senão \(0\)
  \item
    Temos assim uma chave.
  \end{itemize}
\end{enumerate}

\subsection{Resolução do
Exercício}\label{resoluuxe7uxe3o-do-exercuxedcio}

    \begin{Verbatim}[commandchars=\\\{\}]
{\color{incolor}In [{\color{incolor}1}]:} \PY{k+kn}{import} \PY{n+nn}{itertools}\PY{o}{,} \PY{n+nn}{numpy}
        
        \PY{k+kn}{from} \PY{n+nn}{sage.stats.distributions.discrete\PYZus{}gaussian\PYZus{}polynomial} \PY{k+kn}{import} \PY{n}{DiscreteGaussianDistributionPolynomialSampler}
        \PY{k+kn}{from} \PY{n+nn}{sage.modules.free\PYZus{}module\PYZus{}integer} \PY{k+kn}{import} \PY{n}{IntegerLattice}
        \PY{k+kn}{from} \PY{n+nn}{sage.modules.diamond\PYZus{}cutting} \PY{k+kn}{import} \PY{n}{calculate\PYZus{}voronoi\PYZus{}cell}
        
        \PY{n}{dimension} \PY{o}{=} \PY{l+m+mi}{1024}     \PY{c+c1}{\PYZsh{} Grau dos Polinómios (Dimensão suportada pelo NewHope)}
        \PY{n}{modulus} \PY{o}{=} \PY{l+m+mi}{12289}      \PY{c+c1}{\PYZsh{} Módulo (Valor de q)}
        \PY{n}{sigma} \PY{o}{=} \PY{l+m+mi}{8}\PY{o}{/}\PY{n}{sqrt}\PY{p}{(}\PY{l+m+mi}{2}\PY{o}{*}\PY{n}{pi}\PY{p}{)} \PY{c+c1}{\PYZsh{} Sigma (Com o Parâmetro de Distribuição de Ruído = 8)}
        
        \PY{c+c1}{\PYZsh{} Anel Polinomial Quociente}
        \PY{n}{R}\PY{o}{.}\PY{o}{\PYZlt{}}\PY{n}{X}\PY{o}{\PYZgt{}} \PY{o}{=} \PY{n}{PolynomialRing}\PY{p}{(}\PY{n}{GF}\PY{p}{(}\PY{n}{modulus}\PY{p}{)}\PY{p}{)}     \PY{c+c1}{\PYZsh{} Anel polinomial}
        \PY{n}{Y}\PY{o}{.}\PY{o}{\PYZlt{}}\PY{n}{x}\PY{o}{\PYZgt{}} \PY{o}{=} \PY{n}{R}\PY{o}{.}\PY{n}{quotient}\PY{p}{(}\PY{n}{X}\PY{o}{\PYZca{}}\PY{p}{(}\PY{n}{dimension}\PY{p}{)} \PY{o}{+} \PY{l+m+mi}{1}\PY{p}{)}   \PY{c+c1}{\PYZsh{} Campo Ciclotómico}
        
        \PY{n}{subDimension} \PY{o}{=} \PY{l+m+mi}{4}
        
        \PY{c+c1}{\PYZsh{} Função Auxiliar que converte do Intervalo [c0, c1, ..., c1023] para [(c0, c1, c2, c3), ..., (..., c1023)]}
        \PY{k}{def} \PY{n+nf}{grouped}\PY{p}{(}\PY{n}{iterable}\PY{p}{,} \PY{n}{n}\PY{p}{)}\PY{p}{:}
            \PY{k}{return} \PY{n+nb}{zip}\PY{p}{(}\PY{o}{*}\PY{p}{[}\PY{n+nb}{iter}\PY{p}{(}\PY{n}{iterable}\PY{p}{)}\PY{p}{]}\PY{o}{*}\PY{n}{n}\PY{p}{)}
            
        \PY{k}{class} \PY{n+nc}{NewHope}\PY{p}{:}
            
            \PY{c+c1}{\PYZsh{} Inicialização Algortimo}
            \PY{k}{def} \PY{n+nf}{initialize}\PY{p}{(}\PY{n+nb+bp}{self}\PY{p}{)}\PY{p}{:}
        
                \PY{c+c1}{\PYZsh{} Criação da Matriz Identidade}
                \PY{c+c1}{\PYZsh{} Construir uma Rede diagonal de Inteiros a partir da Matriz Identidade}
                \PY{n}{identityMatrix} \PY{o}{=} \PY{n}{Matrix}\PY{o}{.}\PY{n}{identity}\PY{p}{(}\PY{n}{RR}\PY{p}{,} \PY{n}{subDimension}\PY{p}{)}
                \PY{n}{integerLattice} \PY{o}{=} \PY{n}{IntegerLattice}\PY{p}{(}\PY{n}{identityMatrix}\PY{p}{)}
        
                \PY{c+c1}{\PYZsh{} Construir um Half Vector (1/2)}
                \PY{c+c1}{\PYZsh{} Modificar última linha da Matriz Identidade}
                \PY{n}{halfVector} \PY{o}{=} \PY{p}{[}\PY{l+m+mi}{1}\PY{o}{/}\PY{l+m+mi}{2} \PY{k}{for} \PY{n}{i} \PY{o+ow}{in} \PY{n+nb}{range}\PY{p}{(}\PY{n}{subDimension}\PY{p}{)}\PY{p}{]}       
                \PY{n}{identityMatrix}\PY{p}{[}\PY{n}{subDimension} \PY{o}{\PYZhy{}} \PY{l+m+mi}{1}\PY{p}{]} \PY{o}{=} \PY{n}{halfVector}
                
                \PY{c+c1}{\PYZsh{} Criar Célula Voronoi a partir Matriz Modificada}
                \PY{n}{mainPolyhedron} \PY{o}{=} \PY{n}{calculate\PYZus{}voronoi\PYZus{}cell}\PY{p}{(}\PY{n}{identityMatrix}\PY{p}{)}\PY{o}{.}\PY{n}{translation}\PY{p}{(}\PY{n}{halfVector}\PY{p}{)} 
                
                \PY{c+c1}{\PYZsh{} Retorna uma Lattice de Inteiros e um Poliedro de centro em (1/2, ..., 1/2)}
                \PY{k}{return} \PY{p}{(}\PY{n}{integerLattice}\PY{p}{,} \PY{n}{mainPolyhedron}\PY{p}{)}
            
            \PY{k}{def} \PY{n+nf}{dbl}\PY{p}{(}\PY{n+nb+bp}{self}\PY{p}{,} \PY{n}{coefficient\PYZus{}vector}\PY{p}{)}\PY{p}{:}
                \PY{k}{return}  \PY{n}{coefficient\PYZus{}vector} \PY{o}{+} \PYZbs{}
                        \PY{n}{vector}\PY{p}{(} \PY{n}{numpy}\PY{o}{.}\PY{n}{random}\PY{o}{.}\PY{n}{choice}\PY{p}{(}\PY{p}{[}\PY{l+m+mi}{0}\PY{p}{,} \PY{l+m+mi}{1}\PY{p}{]}\PY{p}{,} \PY{n}{p}\PY{o}{=}\PY{p}{[}\PY{l+m+mf}{0.5}\PY{p}{,} \PY{l+m+mf}{0.5}\PY{p}{]}\PY{p}{)} \PY{o}{*} \PY{n}{vector}\PY{p}{(}\PY{p}{[}\PY{l+m+mi}{1}\PY{o}{/}\PY{p}{(}\PY{l+m+mi}{2}\PY{o}{*}\PY{n}{modulus}\PY{p}{)} \PY{k}{for} \PY{n}{\PYZus{}} \PY{o+ow}{in} \PY{n+nb}{range}\PY{p}{(}\PY{n}{subDimension}\PY{p}{)}\PY{p}{]}\PY{p}{)}\PY{p}{)}
        
            \PY{c+c1}{\PYZsh{} Geração do Erro}
            \PY{k}{def} \PY{n+nf}{generateError}\PY{p}{(}\PY{n+nb+bp}{self}\PY{p}{)}\PY{p}{:}
                \PY{n}{f} \PY{o}{=} \PY{n}{DiscreteGaussianDistributionPolynomialSampler}\PY{p}{(}\PY{n}{ZZ}\PY{p}{[}\PY{l+s+s1}{\PYZsq{}}\PY{l+s+s1}{x}\PY{l+s+s1}{\PYZsq{}}\PY{p}{]}\PY{p}{,} \PY{l+m+mi}{5}\PY{p}{,} \PY{n}{sigma}\PY{p}{)}\PY{p}{(}\PY{p}{)}
                \PY{k}{return} \PY{n}{Y}\PY{p}{(}\PY{n}{f}\PY{p}{)}
        
            \PY{c+c1}{\PYZsh{} Geração do Polinómio a partir de Elementos Random}
            \PY{k}{def} \PY{n+nf}{generatePolynomial}\PY{p}{(}\PY{n+nb+bp}{self}\PY{p}{)}\PY{p}{:}
                \PY{k}{return} \PY{n}{Y}\PY{o}{.}\PY{n}{random\PYZus{}element}\PY{p}{(}\PY{p}{)}
        
            \PY{c+c1}{\PYZsh{} Geração do Signal}
            \PY{k}{def} \PY{n+nf}{generateSignal}\PY{p}{(}\PY{n+nb+bp}{self}\PY{p}{,} \PY{n}{poly}\PY{p}{)}\PY{p}{:} 
                
                \PY{c+c1}{\PYZsh{} Divide coeficientes pelo módulo}
                \PY{n}{coefficients} \PY{o}{=} \PY{n+nb}{map}\PY{p}{(}\PY{k}{lambda} \PY{n}{x}\PY{p}{:} \PY{n}{RR}\PY{p}{(}\PY{n}{x}\PY{p}{)} \PY{o}{/} \PY{n}{modulus}\PY{p}{,} \PY{n}{poly}\PY{o}{.}\PY{n}{list}\PY{p}{(}\PY{p}{)}\PY{p}{)}
                \PY{n}{distances} \PY{o}{=} \PY{p}{[}\PY{p}{]}
                
                \PY{k}{for} \PY{n}{v} \PY{o+ow}{in} \PY{n}{grouped}\PY{p}{(}\PY{n}{coefficients}\PY{p}{,} \PY{n}{subDimension}\PY{p}{)}\PY{p}{:}
                    
                    \PY{n}{v} \PY{o}{=} \PY{n+nb+bp}{self}\PY{o}{.}\PY{n}{dbl}\PY{p}{(}\PY{n}{vector}\PY{p}{(}\PY{n}{v}\PY{p}{)}\PY{p}{)}
                    
                    \PY{c+c1}{\PYZsh{} Caso o ponto/vetor esteja no  Main Polyhedron, usa\PYZhy{}se o centro do Main Polyhedron}
                    \PY{k}{if} \PY{n}{mainPolyhedron}\PY{o}{.}\PY{n}{contains}\PY{p}{(}\PY{n}{vector}\PY{p}{(}\PY{n}{v}\PY{p}{)}\PY{p}{)}\PY{p}{:}
                        \PY{n}{distance} \PY{o}{=} \PY{n}{mainPolyhedron}\PY{o}{.}\PY{n}{center}\PY{p}{(}\PY{p}{)} \PY{o}{\PYZhy{}} \PY{n}{v}
                    \PY{c+c1}{\PYZsh{} Caso contrário}
                    \PY{k}{else}\PY{p}{:}
                        \PY{n}{distance} \PY{o}{=} \PY{n}{integerLattice}\PY{o}{.}\PY{n}{closest\PYZus{}vector}\PY{p}{(}\PY{n}{v}\PY{p}{)} \PY{o}{\PYZhy{}} \PY{n}{v}
                    \PY{n}{distances}\PY{o}{.}\PY{n}{append}\PY{p}{(}\PY{n}{distance}\PY{p}{)}
        
                \PY{k}{return} \PY{n}{distances}
            
            \PY{c+c1}{\PYZsh{} Reconciliação}
            \PY{k}{def} \PY{n+nf}{reconcile}\PY{p}{(}\PY{n+nb+bp}{self}\PY{p}{,} \PY{n}{poly}\PY{p}{,} \PY{n}{w}\PY{p}{)}\PY{p}{:}
                
                \PY{n}{coefficients} \PY{o}{=} \PY{n+nb}{map}\PY{p}{(}\PY{k}{lambda} \PY{n}{x}\PY{p}{:} \PY{n}{RR}\PY{p}{(}\PY{n}{x}\PY{p}{)} \PY{o}{/} \PY{n}{modulus}\PY{p}{,} \PY{n}{poly}\PY{o}{.}\PY{n}{list}\PY{p}{(}\PY{p}{)}\PY{p}{)}
                \PY{n}{key} \PY{o}{=} \PY{p}{[}\PY{p}{]}
                
                \PY{k}{for} \PY{n}{difference}\PY{p}{,} \PY{n}{v} \PY{o+ow}{in} \PY{n+nb}{zip}\PY{p}{(}\PY{n}{w}\PY{p}{,} \PY{n}{grouped}\PY{p}{(}\PY{n}{coefficients}\PY{p}{,} \PY{n}{subDimension}\PY{p}{)}\PY{p}{)}\PY{p}{:}
                    
                    \PY{n}{v} \PY{o}{=} \PY{n+nb+bp}{self}\PY{o}{.}\PY{n}{dbl}\PY{p}{(}\PY{n}{vector}\PY{p}{(}\PY{n}{v}\PY{p}{)}\PY{p}{)}
                    \PY{n}{coordinate} \PY{o}{=} \PY{n}{vector}\PY{p}{(}\PY{p}{[}\PY{n+nb}{round}\PY{p}{(}\PY{n}{point}\PY{p}{,} \PY{l+m+mi}{1}\PY{p}{)} \PY{k}{for} \PY{n}{point} \PY{o+ow}{in} \PY{p}{(}\PY{n}{v} \PY{o}{+} \PY{n}{difference}\PY{p}{)} \PY{p}{]}\PY{p}{)}
                    \PY{n}{key}\PY{o}{.}\PY{n}{append}\PY{p}{(}\PY{l+m+mi}{1} \PY{k}{if} \PY{n}{coordinate} \PY{o}{==} \PY{n}{mainPolyhedron}\PY{o}{.}\PY{n}{center}\PY{p}{(}\PY{p}{)} \PY{k}{else} \PY{l+m+mi}{0}\PY{p}{)}
        
                \PY{k}{return} \PY{l+s+s2}{\PYZdq{}}\PY{l+s+s2}{\PYZdq{}}\PY{o}{.}\PY{n}{join}\PY{p}{(}\PY{n+nb}{map}\PY{p}{(}\PY{n+nb}{str}\PY{p}{,} \PY{n}{key}\PY{p}{)}\PY{p}{)}
\end{Verbatim}

\vspace{2mm}

    \subsection{Teste da Classe}\label{teste-da-classe}
\vspace{2mm}

\textbf{Para teste da classe e algoritmo total fez-se:}
\vspace{2mm}

\begin{itemize}
\tightlist
\item
  Inicialização do anel diagonal e o poliedro
\item
  Cria-se uma matriz partilhada para os dois
\item
  Instanciação dos valores do Bob (segredo, erro e o valor)
\item
  Instanciação dos valores da Alice (segredo, erro e o valor)
\item
  A chave da Alice sendo: \$ valueBob * secretAlice + temp\_error\$
\item
  A chave do Bob sendo: \$ valueAlice * secretBob \$
\item
  Fazer a reconciliação destes polinómios e ver se são iguais.
\end{itemize}
\vspace{3mm}

    \textbf{Teste e Verificação para o modo PKE-IND-CCA}

    \begin{Verbatim}[commandchars=\\\{\}]
{\color{incolor}In [{\color{incolor}2}]:} \PY{c+c1}{\PYZsh{} Criação da Instância}
        \PY{n}{newHope} \PY{o}{=} \PY{n}{NewHope}\PY{p}{(}\PY{p}{)}
        
        \PY{c+c1}{\PYZsh{} Lattice de Inteiros e Poliedro Principal}
        \PY{p}{(}\PY{n}{integerLattice}\PY{p}{,} \PY{n}{mainPolyhedron}\PY{p}{)} \PY{o}{=} \PY{n}{newHope}\PY{o}{.}\PY{n}{initialize}\PY{p}{(}\PY{p}{)}
        
        \PY{c+c1}{\PYZsh{} Matriz Partilhada}
        \PY{n}{shared} \PY{o}{=} \PY{n}{newHope}\PY{o}{.}\PY{n}{generatePolynomial}\PY{p}{(}\PY{p}{)}
        
        \PY{c+c1}{\PYZsh{} 1. Valores Bob}
        \PY{n}{secretBob} \PY{o}{=} \PY{n}{newHope}\PY{o}{.}\PY{n}{generateError}\PY{p}{(}\PY{p}{)}
        \PY{n}{errorBob} \PY{o}{=} \PY{n}{newHope}\PY{o}{.}\PY{n}{generateError}\PY{p}{(}\PY{p}{)}
        \PY{n}{valueBob} \PY{o}{=} \PY{n}{shared} \PY{o}{*} \PY{n}{secretBob} \PY{o}{+} \PY{n}{errorBob}
        
        \PY{c+c1}{\PYZsh{} 2. Valores Alice}
        \PY{n}{secretAlice} \PY{o}{=} \PY{n}{newHope}\PY{o}{.}\PY{n}{generateError}\PY{p}{(}\PY{p}{)}
        \PY{n}{errorAlice} \PY{o}{=} \PY{n}{newHope}\PY{o}{.}\PY{n}{generateError}\PY{p}{(}\PY{p}{)}
        \PY{n}{valueAlice} \PY{o}{=} \PY{n}{shared} \PY{o}{*} \PY{n}{secretAlice} \PY{o}{+} \PY{n}{errorAlice}
        
        \PY{c+c1}{\PYZsh{} Key Alice}
        \PY{n}{temp\PYZus{}error} \PY{o}{=} \PY{n}{newHope}\PY{o}{.}\PY{n}{generateError}\PY{p}{(}\PY{p}{)}
        \PY{n}{keyAlice} \PY{o}{=} \PY{n}{valueBob} \PY{o}{*} \PY{n}{secretAlice} \PY{o}{+} \PY{n}{temp\PYZus{}error}
        \PY{n}{w} \PY{o}{=} \PY{n}{newHope}\PY{o}{.}\PY{n}{generateSignal}\PY{p}{(}\PY{n}{keyAlice}\PY{p}{)}
        \PY{n}{keyBinaryAlice} \PY{o}{=} \PY{n}{newHope}\PY{o}{.}\PY{n}{reconcile}\PY{p}{(}\PY{n}{keyAlice}\PY{p}{,} \PY{n}{w}\PY{p}{)}
        
        \PY{c+c1}{\PYZsh{} Key Bob}
        \PY{n}{keyBob} \PY{o}{=} \PY{n}{valueAlice} \PY{o}{*} \PY{n}{secretBob}
        \PY{n}{keyBinaryBob} \PY{o}{=} \PY{n}{newHope}\PY{o}{.}\PY{n}{reconcile}\PY{p}{(}\PY{n}{keyBob}\PY{p}{,} \PY{n}{w}\PY{p}{)}
        
        \PY{k}{if} \PY{p}{(}\PY{n}{keyBinaryBob} \PY{o}{==} \PY{n}{keyBinaryAlice}\PY{p}{)}\PY{p}{:}
            \PY{k}{print} \PY{l+s+s2}{\PYZdq{}}\PY{l+s+s2}{As chaves correspondem.}\PY{l+s+s2}{\PYZdq{}}
            \PY{k}{print} \PY{n+nb}{hex}\PY{p}{(}\PY{n+nb}{int}\PY{p}{(}\PY{n}{keyBinaryBob}\PY{p}{,} \PY{l+m+mi}{2}\PY{p}{)}\PY{p}{)}
        \PY{k}{else}\PY{p}{:}
            \PY{k}{print} \PY{l+s+s2}{\PYZdq{}}\PY{l+s+s2}{Erro na correspondência das chaves.}\PY{l+s+s2}{\PYZdq{}}
\end{Verbatim}


    \begin{Verbatim}[commandchars=\\\{\}]
As chaves correspondem.
0x6407f9782c69ab7ee51a61a4235109af87ee0d0f40cc1d9e8899e1c0050e2d0L

    \end{Verbatim}

    \subsection{Observações Finais}\label{observauxe7uxf5es-finais}
\vspace{2mm}

\begin{itemize}
\tightlist
\item
  O algoritmo \textbf{NewHope} foi muito complicado de entender, dado que o próprio documento principal de envio não continha o algoritmo propriamente descrito, tal como acontecia para o \textbf{NTRU-Prime};
\item
Recorreu-se à implementação deste algoritmo realizado através de outras linguagens de programação (incluindo o próprio Python), de forma a tentar compreender todo o processo algorítmico por detrás.
\end{itemize}

    \subsection{Referências}\label{referuxeancias}
\vspace{2mm}

\begin{itemize}
\tightlist
\item
  GitHub, New Hope Key Exchange and Key Encapsulation \url{https://github.com/Art3misOne/newhope} (Acedido a 10 maio
  2020)
\item
 GitHub, PyNewHope \url{https://github.com/scottwn/PyNewHope} (Acedido a 10 maio 2020)
\end{itemize}

    
\end{document}
