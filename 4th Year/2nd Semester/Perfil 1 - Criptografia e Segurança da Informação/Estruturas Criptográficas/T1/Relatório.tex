\documentclass[12pt]{report}
\renewcommand{\thesection}{\arabic{section}}
\usepackage[portuguese]{babel}

    \usepackage[breakable]{tcolorbox}
    \usepackage{parskip} % Stop auto-indenting (to mimic markdown behaviour)
    
    \usepackage{iftex}
    \ifPDFTeX
    	\usepackage[T1]{fontenc}
    	\usepackage{mathpazo}
    \else
    	\usepackage{fontspec}
    \fi

    % Basic figure setup, for now with no caption control since it's done
    % automatically by Pandoc (which extracts ![](path) syntax from Markdown).
    \usepackage{graphicx}
    % Maintain compatibility with old templates. Remove in nbconvert 6.0
    \let\Oldincludegraphics\includegraphics
    % Ensure that by default, figures have no caption (until we provide a
    % proper Figure object with a Caption API and a way to capture that
    % in the conversion process - todo).
    \usepackage{caption}
    \DeclareCaptionFormat{nocaption}{}
    \captionsetup{format=nocaption,aboveskip=0pt,belowskip=0pt}

    \usepackage[Export]{adjustbox} % Used to constrain images to a maximum size
    \adjustboxset{max size={0.9\linewidth}{0.9\paperheight}}
    \usepackage{float}
    \floatplacement{figure}{H} % forces figures to be placed at the correct location
    \usepackage{xcolor} % Allow colors to be defined
    \usepackage{enumerate} % Needed for markdown enumerations to work
    \usepackage{geometry} % Used to adjust the document margins
    \usepackage{amsmath} % Equations
    \usepackage{amssymb} % Equations
    \usepackage{textcomp} % defines textquotesingle
    % Hack from http://tex.stackexchange.com/a/47451/13684:
    \AtBeginDocument{%
        \def\PYZsq{\textquotesingle}% Upright quotes in Pygmentized code
    }
    \usepackage{upquote} % Upright quotes for verbatim code
    \usepackage{eurosym} % defines \euro
    \usepackage[mathletters]{ucs} % Extended unicode (utf-8) support
    \usepackage{fancyvrb} % verbatim replacement that allows latex
    \usepackage{grffile} % extends the file name processing of package graphics 
                         % to support a larger range
    \makeatletter % fix for grffile with XeLaTeX
    \def\Gread@@xetex#1{%
      \IfFileExists{"\Gin@base".bb}%
      {\Gread@eps{\Gin@base.bb}}%
      {\Gread@@xetex@aux#1}%
    }
    \makeatother

    % The hyperref package gives us a pdf with properly built
    % internal navigation ('pdf bookmarks' for the table of contents,
    % internal cross-reference links, web links for URLs, etc.)
    \usepackage{hyperref}
    % The default LaTeX title has an obnoxious amount of whitespace. By default,
    % titling removes some of it. It also provides customization options.
    \usepackage{titling}
    \usepackage{longtable} % longtable support required by pandoc >1.10
    \usepackage{booktabs}  % table support for pandoc > 1.12.2
    \usepackage[inline]{enumitem} % IRkernel/repr support (it uses the enumerate* environment)
    \usepackage[normalem]{ulem} % ulem is needed to support strikethroughs (\sout)
                                % normalem makes italics be italics, not underlines
    \usepackage{mathrsfs}
    

    
    % Colors for the hyperref package
    \definecolor{urlcolor}{rgb}{0,.145,.698}
    \definecolor{linkcolor}{rgb}{.71,0.21,0.01}
    \definecolor{citecolor}{rgb}{.12,.54,.11}

    % ANSI colors
    \definecolor{ansi-black}{HTML}{3E424D}
    \definecolor{ansi-black-intense}{HTML}{282C36}
    \definecolor{ansi-red}{HTML}{E75C58}
    \definecolor{ansi-red-intense}{HTML}{B22B31}
    \definecolor{ansi-green}{HTML}{00A250}
    \definecolor{ansi-green-intense}{HTML}{007427}
    \definecolor{ansi-yellow}{HTML}{DDB62B}
    \definecolor{ansi-yellow-intense}{HTML}{B27D12}
    \definecolor{ansi-blue}{HTML}{208FFB}
    \definecolor{ansi-blue-intense}{HTML}{0065CA}
    \definecolor{ansi-magenta}{HTML}{D160C4}
    \definecolor{ansi-magenta-intense}{HTML}{A03196}
    \definecolor{ansi-cyan}{HTML}{60C6C8}
    \definecolor{ansi-cyan-intense}{HTML}{258F8F}
    \definecolor{ansi-white}{HTML}{C5C1B4}
    \definecolor{ansi-white-intense}{HTML}{A1A6B2}
    \definecolor{ansi-default-inverse-fg}{HTML}{FFFFFF}
    \definecolor{ansi-default-inverse-bg}{HTML}{000000}

    % commands and environments needed by pandoc snippets
    % extracted from the output of `pandoc -s`
    \providecommand{\tightlist}{%
      \setlength{\itemsep}{0pt}\setlength{\parskip}{0pt}}
    \DefineVerbatimEnvironment{Highlighting}{Verbatim}{commandchars=\\\{\}}
    % Add ',fontsize=\small' for more characters per line
    \newenvironment{Shaded}{}{}
    \newcommand{\KeywordTok}[1]{\textcolor[rgb]{0.00,0.44,0.13}{\textbf{{#1}}}}
    \newcommand{\DataTypeTok}[1]{\textcolor[rgb]{0.56,0.13,0.00}{{#1}}}
    \newcommand{\DecValTok}[1]{\textcolor[rgb]{0.25,0.63,0.44}{{#1}}}
    \newcommand{\BaseNTok}[1]{\textcolor[rgb]{0.25,0.63,0.44}{{#1}}}
    \newcommand{\FloatTok}[1]{\textcolor[rgb]{0.25,0.63,0.44}{{#1}}}
    \newcommand{\CharTok}[1]{\textcolor[rgb]{0.25,0.44,0.63}{{#1}}}
    \newcommand{\StringTok}[1]{\textcolor[rgb]{0.25,0.44,0.63}{{#1}}}
    \newcommand{\CommentTok}[1]{\textcolor[rgb]{0.38,0.63,0.69}{\textit{{#1}}}}
    \newcommand{\OtherTok}[1]{\textcolor[rgb]{0.00,0.44,0.13}{{#1}}}
    \newcommand{\AlertTok}[1]{\textcolor[rgb]{1.00,0.00,0.00}{\textbf{{#1}}}}
    \newcommand{\FunctionTok}[1]{\textcolor[rgb]{0.02,0.16,0.49}{{#1}}}
    \newcommand{\RegionMarkerTok}[1]{{#1}}
    \newcommand{\ErrorTok}[1]{\textcolor[rgb]{1.00,0.00,0.00}{\textbf{{#1}}}}
    \newcommand{\NormalTok}[1]{{#1}}
    
    % Additional commands for more recent versions of Pandoc
    \newcommand{\ConstantTok}[1]{\textcolor[rgb]{0.53,0.00,0.00}{{#1}}}
    \newcommand{\SpecialCharTok}[1]{\textcolor[rgb]{0.25,0.44,0.63}{{#1}}}
    \newcommand{\VerbatimStringTok}[1]{\textcolor[rgb]{0.25,0.44,0.63}{{#1}}}
    \newcommand{\SpecialStringTok}[1]{\textcolor[rgb]{0.73,0.40,0.53}{{#1}}}
    \newcommand{\ImportTok}[1]{{#1}}
    \newcommand{\DocumentationTok}[1]{\textcolor[rgb]{0.73,0.13,0.13}{\textit{{#1}}}}
    \newcommand{\AnnotationTok}[1]{\textcolor[rgb]{0.38,0.63,0.69}{\textbf{\textit{{#1}}}}}
    \newcommand{\CommentVarTok}[1]{\textcolor[rgb]{0.38,0.63,0.69}{\textbf{\textit{{#1}}}}}
    \newcommand{\VariableTok}[1]{\textcolor[rgb]{0.10,0.09,0.49}{{#1}}}
    \newcommand{\ControlFlowTok}[1]{\textcolor[rgb]{0.00,0.44,0.13}{\textbf{{#1}}}}
    \newcommand{\OperatorTok}[1]{\textcolor[rgb]{0.40,0.40,0.40}{{#1}}}
    \newcommand{\BuiltInTok}[1]{{#1}}
    \newcommand{\ExtensionTok}[1]{{#1}}
    \newcommand{\PreprocessorTok}[1]{\textcolor[rgb]{0.74,0.48,0.00}{{#1}}}
    \newcommand{\AttributeTok}[1]{\textcolor[rgb]{0.49,0.56,0.16}{{#1}}}
    \newcommand{\InformationTok}[1]{\textcolor[rgb]{0.38,0.63,0.69}{\textbf{\textit{{#1}}}}}
    \newcommand{\WarningTok}[1]{\textcolor[rgb]{0.38,0.63,0.69}{\textbf{\textit{{#1}}}}}
    
    
    % Define a nice break command that doesn't care if a line doesn't already
    % exist.
    \def\br{\hspace*{\fill} \\* }
    % Math Jax compatibility definitions
    \def\gt{>}
    \def\lt{<}
    \let\Oldtex\TeX
    \let\Oldlatex\LaTeX
    \renewcommand{\TeX}{\textrm{\Oldtex}}
    \renewcommand{\LaTeX}{\textrm{\Oldlatex}}
    % Document parameters
    % Document title
    \title{T1E01}
    
    
    
    
    
% Pygments definitions
\makeatletter
\def\PY@reset{\let\PY@it=\relax \let\PY@bf=\relax%
    \let\PY@ul=\relax \let\PY@tc=\relax%
    \let\PY@bc=\relax \let\PY@ff=\relax}
\def\PY@tok#1{\csname PY@tok@#1\endcsname}
\def\PY@toks#1+{\ifx\relax#1\empty\else%
    \PY@tok{#1}\expandafter\PY@toks\fi}
\def\PY@do#1{\PY@bc{\PY@tc{\PY@ul{%
    \PY@it{\PY@bf{\PY@ff{#1}}}}}}}
\def\PY#1#2{\PY@reset\PY@toks#1+\relax+\PY@do{#2}}

\expandafter\def\csname PY@tok@w\endcsname{\def\PY@tc##1{\textcolor[rgb]{0.73,0.73,0.73}{##1}}}
\expandafter\def\csname PY@tok@c\endcsname{\let\PY@it=\textit\def\PY@tc##1{\textcolor[rgb]{0.25,0.50,0.50}{##1}}}
\expandafter\def\csname PY@tok@cp\endcsname{\def\PY@tc##1{\textcolor[rgb]{0.74,0.48,0.00}{##1}}}
\expandafter\def\csname PY@tok@k\endcsname{\let\PY@bf=\textbf\def\PY@tc##1{\textcolor[rgb]{0.00,0.50,0.00}{##1}}}
\expandafter\def\csname PY@tok@kp\endcsname{\def\PY@tc##1{\textcolor[rgb]{0.00,0.50,0.00}{##1}}}
\expandafter\def\csname PY@tok@kt\endcsname{\def\PY@tc##1{\textcolor[rgb]{0.69,0.00,0.25}{##1}}}
\expandafter\def\csname PY@tok@o\endcsname{\def\PY@tc##1{\textcolor[rgb]{0.40,0.40,0.40}{##1}}}
\expandafter\def\csname PY@tok@ow\endcsname{\let\PY@bf=\textbf\def\PY@tc##1{\textcolor[rgb]{0.67,0.13,1.00}{##1}}}
\expandafter\def\csname PY@tok@nb\endcsname{\def\PY@tc##1{\textcolor[rgb]{0.00,0.50,0.00}{##1}}}
\expandafter\def\csname PY@tok@nf\endcsname{\def\PY@tc##1{\textcolor[rgb]{0.00,0.00,1.00}{##1}}}
\expandafter\def\csname PY@tok@nc\endcsname{\let\PY@bf=\textbf\def\PY@tc##1{\textcolor[rgb]{0.00,0.00,1.00}{##1}}}
\expandafter\def\csname PY@tok@nn\endcsname{\let\PY@bf=\textbf\def\PY@tc##1{\textcolor[rgb]{0.00,0.00,1.00}{##1}}}
\expandafter\def\csname PY@tok@ne\endcsname{\let\PY@bf=\textbf\def\PY@tc##1{\textcolor[rgb]{0.82,0.25,0.23}{##1}}}
\expandafter\def\csname PY@tok@nv\endcsname{\def\PY@tc##1{\textcolor[rgb]{0.10,0.09,0.49}{##1}}}
\expandafter\def\csname PY@tok@no\endcsname{\def\PY@tc##1{\textcolor[rgb]{0.53,0.00,0.00}{##1}}}
\expandafter\def\csname PY@tok@nl\endcsname{\def\PY@tc##1{\textcolor[rgb]{0.63,0.63,0.00}{##1}}}
\expandafter\def\csname PY@tok@ni\endcsname{\let\PY@bf=\textbf\def\PY@tc##1{\textcolor[rgb]{0.60,0.60,0.60}{##1}}}
\expandafter\def\csname PY@tok@na\endcsname{\def\PY@tc##1{\textcolor[rgb]{0.49,0.56,0.16}{##1}}}
\expandafter\def\csname PY@tok@nt\endcsname{\let\PY@bf=\textbf\def\PY@tc##1{\textcolor[rgb]{0.00,0.50,0.00}{##1}}}
\expandafter\def\csname PY@tok@nd\endcsname{\def\PY@tc##1{\textcolor[rgb]{0.67,0.13,1.00}{##1}}}
\expandafter\def\csname PY@tok@s\endcsname{\def\PY@tc##1{\textcolor[rgb]{0.73,0.13,0.13}{##1}}}
\expandafter\def\csname PY@tok@sd\endcsname{\let\PY@it=\textit\def\PY@tc##1{\textcolor[rgb]{0.73,0.13,0.13}{##1}}}
\expandafter\def\csname PY@tok@si\endcsname{\let\PY@bf=\textbf\def\PY@tc##1{\textcolor[rgb]{0.73,0.40,0.53}{##1}}}
\expandafter\def\csname PY@tok@se\endcsname{\let\PY@bf=\textbf\def\PY@tc##1{\textcolor[rgb]{0.73,0.40,0.13}{##1}}}
\expandafter\def\csname PY@tok@sr\endcsname{\def\PY@tc##1{\textcolor[rgb]{0.73,0.40,0.53}{##1}}}
\expandafter\def\csname PY@tok@ss\endcsname{\def\PY@tc##1{\textcolor[rgb]{0.10,0.09,0.49}{##1}}}
\expandafter\def\csname PY@tok@sx\endcsname{\def\PY@tc##1{\textcolor[rgb]{0.00,0.50,0.00}{##1}}}
\expandafter\def\csname PY@tok@m\endcsname{\def\PY@tc##1{\textcolor[rgb]{0.40,0.40,0.40}{##1}}}
\expandafter\def\csname PY@tok@gh\endcsname{\let\PY@bf=\textbf\def\PY@tc##1{\textcolor[rgb]{0.00,0.00,0.50}{##1}}}
\expandafter\def\csname PY@tok@gu\endcsname{\let\PY@bf=\textbf\def\PY@tc##1{\textcolor[rgb]{0.50,0.00,0.50}{##1}}}
\expandafter\def\csname PY@tok@gd\endcsname{\def\PY@tc##1{\textcolor[rgb]{0.63,0.00,0.00}{##1}}}
\expandafter\def\csname PY@tok@gi\endcsname{\def\PY@tc##1{\textcolor[rgb]{0.00,0.63,0.00}{##1}}}
\expandafter\def\csname PY@tok@gr\endcsname{\def\PY@tc##1{\textcolor[rgb]{1.00,0.00,0.00}{##1}}}
\expandafter\def\csname PY@tok@ge\endcsname{\let\PY@it=\textit}
\expandafter\def\csname PY@tok@gs\endcsname{\let\PY@bf=\textbf}
\expandafter\def\csname PY@tok@gp\endcsname{\let\PY@bf=\textbf\def\PY@tc##1{\textcolor[rgb]{0.00,0.00,0.50}{##1}}}
\expandafter\def\csname PY@tok@go\endcsname{\def\PY@tc##1{\textcolor[rgb]{0.53,0.53,0.53}{##1}}}
\expandafter\def\csname PY@tok@gt\endcsname{\def\PY@tc##1{\textcolor[rgb]{0.00,0.27,0.87}{##1}}}
\expandafter\def\csname PY@tok@err\endcsname{\def\PY@bc##1{\setlength{\fboxsep}{0pt}\fcolorbox[rgb]{1.00,0.00,0.00}{1,1,1}{\strut ##1}}}
\expandafter\def\csname PY@tok@kc\endcsname{\let\PY@bf=\textbf\def\PY@tc##1{\textcolor[rgb]{0.00,0.50,0.00}{##1}}}
\expandafter\def\csname PY@tok@kd\endcsname{\let\PY@bf=\textbf\def\PY@tc##1{\textcolor[rgb]{0.00,0.50,0.00}{##1}}}
\expandafter\def\csname PY@tok@kn\endcsname{\let\PY@bf=\textbf\def\PY@tc##1{\textcolor[rgb]{0.00,0.50,0.00}{##1}}}
\expandafter\def\csname PY@tok@kr\endcsname{\let\PY@bf=\textbf\def\PY@tc##1{\textcolor[rgb]{0.00,0.50,0.00}{##1}}}
\expandafter\def\csname PY@tok@bp\endcsname{\def\PY@tc##1{\textcolor[rgb]{0.00,0.50,0.00}{##1}}}
\expandafter\def\csname PY@tok@fm\endcsname{\def\PY@tc##1{\textcolor[rgb]{0.00,0.00,1.00}{##1}}}
\expandafter\def\csname PY@tok@vc\endcsname{\def\PY@tc##1{\textcolor[rgb]{0.10,0.09,0.49}{##1}}}
\expandafter\def\csname PY@tok@vg\endcsname{\def\PY@tc##1{\textcolor[rgb]{0.10,0.09,0.49}{##1}}}
\expandafter\def\csname PY@tok@vi\endcsname{\def\PY@tc##1{\textcolor[rgb]{0.10,0.09,0.49}{##1}}}
\expandafter\def\csname PY@tok@vm\endcsname{\def\PY@tc##1{\textcolor[rgb]{0.10,0.09,0.49}{##1}}}
\expandafter\def\csname PY@tok@sa\endcsname{\def\PY@tc##1{\textcolor[rgb]{0.73,0.13,0.13}{##1}}}
\expandafter\def\csname PY@tok@sb\endcsname{\def\PY@tc##1{\textcolor[rgb]{0.73,0.13,0.13}{##1}}}
\expandafter\def\csname PY@tok@sc\endcsname{\def\PY@tc##1{\textcolor[rgb]{0.73,0.13,0.13}{##1}}}
\expandafter\def\csname PY@tok@dl\endcsname{\def\PY@tc##1{\textcolor[rgb]{0.73,0.13,0.13}{##1}}}
\expandafter\def\csname PY@tok@s2\endcsname{\def\PY@tc##1{\textcolor[rgb]{0.73,0.13,0.13}{##1}}}
\expandafter\def\csname PY@tok@sh\endcsname{\def\PY@tc##1{\textcolor[rgb]{0.73,0.13,0.13}{##1}}}
\expandafter\def\csname PY@tok@s1\endcsname{\def\PY@tc##1{\textcolor[rgb]{0.73,0.13,0.13}{##1}}}
\expandafter\def\csname PY@tok@mb\endcsname{\def\PY@tc##1{\textcolor[rgb]{0.40,0.40,0.40}{##1}}}
\expandafter\def\csname PY@tok@mf\endcsname{\def\PY@tc##1{\textcolor[rgb]{0.40,0.40,0.40}{##1}}}
\expandafter\def\csname PY@tok@mh\endcsname{\def\PY@tc##1{\textcolor[rgb]{0.40,0.40,0.40}{##1}}}
\expandafter\def\csname PY@tok@mi\endcsname{\def\PY@tc##1{\textcolor[rgb]{0.40,0.40,0.40}{##1}}}
\expandafter\def\csname PY@tok@il\endcsname{\def\PY@tc##1{\textcolor[rgb]{0.40,0.40,0.40}{##1}}}
\expandafter\def\csname PY@tok@mo\endcsname{\def\PY@tc##1{\textcolor[rgb]{0.40,0.40,0.40}{##1}}}
\expandafter\def\csname PY@tok@ch\endcsname{\let\PY@it=\textit\def\PY@tc##1{\textcolor[rgb]{0.25,0.50,0.50}{##1}}}
\expandafter\def\csname PY@tok@cm\endcsname{\let\PY@it=\textit\def\PY@tc##1{\textcolor[rgb]{0.25,0.50,0.50}{##1}}}
\expandafter\def\csname PY@tok@cpf\endcsname{\let\PY@it=\textit\def\PY@tc##1{\textcolor[rgb]{0.25,0.50,0.50}{##1}}}
\expandafter\def\csname PY@tok@c1\endcsname{\let\PY@it=\textit\def\PY@tc##1{\textcolor[rgb]{0.25,0.50,0.50}{##1}}}
\expandafter\def\csname PY@tok@cs\endcsname{\let\PY@it=\textit\def\PY@tc##1{\textcolor[rgb]{0.25,0.50,0.50}{##1}}}

\def\PYZbs{\char`\\}
\def\PYZus{\char`\_}
\def\PYZob{\char`\{}
\def\PYZcb{\char`\}}
\def\PYZca{\char`\^}
\def\PYZam{\char`\&}
\def\PYZlt{\char`\<}
\def\PYZgt{\char`\>}
\def\PYZsh{\char`\#}
\def\PYZpc{\char`\%}
\def\PYZdl{\char`\$}
\def\PYZhy{\char`\-}
\def\PYZsq{\char`\'}
\def\PYZdq{\char`\"}
\def\PYZti{\char`\~}
% for compatibility with earlier versions
\def\PYZat{@}
\def\PYZlb{[}
\def\PYZrb{]}
\makeatother


    % For linebreaks inside Verbatim environment from package fancyvrb. 
    \makeatletter
        \newbox\Wrappedcontinuationbox 
        \newbox\Wrappedvisiblespacebox 
        \newcommand*\Wrappedvisiblespace {\textcolor{red}{\textvisiblespace}} 
        \newcommand*\Wrappedcontinuationsymbol {\textcolor{red}{\llap{\tiny$\m@th\hookrightarrow$}}} 
        \newcommand*\Wrappedcontinuationindent {3ex } 
        \newcommand*\Wrappedafterbreak {\kern\Wrappedcontinuationindent\copy\Wrappedcontinuationbox} 
        % Take advantage of the already applied Pygments mark-up to insert 
        % potential linebreaks for TeX processing. 
        %        {, <, #, %, $, ' and ": go to next line. 
        %        _, }, ^, &, >, - and ~: stay at end of broken line. 
        % Use of \textquotesingle for straight quote. 
        \newcommand*\Wrappedbreaksatspecials {% 
            \def\PYGZus{\discretionary{\char`\_}{\Wrappedafterbreak}{\char`\_}}% 
            \def\PYGZob{\discretionary{}{\Wrappedafterbreak\char`\{}{\char`\{}}% 
            \def\PYGZcb{\discretionary{\char`\}}{\Wrappedafterbreak}{\char`\}}}% 
            \def\PYGZca{\discretionary{\char`\^}{\Wrappedafterbreak}{\char`\^}}% 
            \def\PYGZam{\discretionary{\char`\&}{\Wrappedafterbreak}{\char`\&}}% 
            \def\PYGZlt{\discretionary{}{\Wrappedafterbreak\char`\<}{\char`\<}}% 
            \def\PYGZgt{\discretionary{\char`\>}{\Wrappedafterbreak}{\char`\>}}% 
            \def\PYGZsh{\discretionary{}{\Wrappedafterbreak\char`\#}{\char`\#}}% 
            \def\PYGZpc{\discretionary{}{\Wrappedafterbreak\char`\%}{\char`\%}}% 
            \def\PYGZdl{\discretionary{}{\Wrappedafterbreak\char`\$}{\char`\$}}% 
            \def\PYGZhy{\discretionary{\char`\-}{\Wrappedafterbreak}{\char`\-}}% 
            \def\PYGZsq{\discretionary{}{\Wrappedafterbreak\textquotesingle}{\textquotesingle}}% 
            \def\PYGZdq{\discretionary{}{\Wrappedafterbreak\char`\"}{\char`\"}}% 
            \def\PYGZti{\discretionary{\char`\~}{\Wrappedafterbreak}{\char`\~}}% 
        } 
        % Some characters . , ; ? ! / are not pygmentized. 
        % This macro makes them "active" and they will insert potential linebreaks 
        \newcommand*\Wrappedbreaksatpunct {% 
            \lccode`\~`\.\lowercase{\def~}{\discretionary{\hbox{\char`\.}}{\Wrappedafterbreak}{\hbox{\char`\.}}}% 
            \lccode`\~`\,\lowercase{\def~}{\discretionary{\hbox{\char`\,}}{\Wrappedafterbreak}{\hbox{\char`\,}}}% 
            \lccode`\~`\;\lowercase{\def~}{\discretionary{\hbox{\char`\;}}{\Wrappedafterbreak}{\hbox{\char`\;}}}% 
            \lccode`\~`\:\lowercase{\def~}{\discretionary{\hbox{\char`\:}}{\Wrappedafterbreak}{\hbox{\char`\:}}}% 
            \lccode`\~`\?\lowercase{\def~}{\discretionary{\hbox{\char`\?}}{\Wrappedafterbreak}{\hbox{\char`\?}}}% 
            \lccode`\~`\!\lowercase{\def~}{\discretionary{\hbox{\char`\!}}{\Wrappedafterbreak}{\hbox{\char`\!}}}% 
            \lccode`\~`\/\lowercase{\def~}{\discretionary{\hbox{\char`\/}}{\Wrappedafterbreak}{\hbox{\char`\/}}}% 
            \catcode`\.\active
            \catcode`\,\active 
            \catcode`\;\active
            \catcode`\:\active
            \catcode`\?\active
            \catcode`\!\active
            \catcode`\/\active 
            \lccode`\~`\~ 	
        }
    \makeatother

    \let\OriginalVerbatim=\Verbatim
    \makeatletter
    \renewcommand{\Verbatim}[1][1]{%
        %\parskip\z@skip
        \sbox\Wrappedcontinuationbox {\Wrappedcontinuationsymbol}%
        \sbox\Wrappedvisiblespacebox {\FV@SetupFont\Wrappedvisiblespace}%
        \def\FancyVerbFormatLine ##1{\hsize\linewidth
            \vtop{\raggedright\hyphenpenalty\z@\exhyphenpenalty\z@
                \doublehyphendemerits\z@\finalhyphendemerits\z@
                \strut ##1\strut}%
        }%
        % If the linebreak is at a space, the latter will be displayed as visible
        % space at end of first line, and a continuation symbol starts next line.
        % Stretch/shrink are however usually zero for typewriter font.
        \def\FV@Space {%
            \nobreak\hskip\z@ plus\fontdimen3\font minus\fontdimen4\font
            \discretionary{\copy\Wrappedvisiblespacebox}{\Wrappedafterbreak}
            {\kern\fontdimen2\font}%
        }%
        
        % Allow breaks at special characters using \PYG... macros.
        \Wrappedbreaksatspecials
        % Breaks at punctuation characters . , ; ? ! and / need catcode=\active 	
        \OriginalVerbatim[#1,codes*=\Wrappedbreaksatpunct]%
    }
    \makeatother

    % Exact colors from NB
    \definecolor{incolor}{HTML}{303F9F}
    \definecolor{outcolor}{HTML}{D84315}
    \definecolor{cellborder}{HTML}{CFCFCF}
    \definecolor{cellbackground}{HTML}{F7F7F7}
    
    % prompt
    \makeatletter
    \newcommand{\boxspacing}{\kern\kvtcb@left@rule\kern\kvtcb@boxsep}
    \makeatother
    \newcommand{\prompt}[4]{
        \ttfamily\llap{{\color{#2}[#3]:\hspace{3pt}#4}}\vspace{-\baselineskip}
    }
    

    
    % Prevent overflowing lines due to hard-to-break entities
    \sloppy 
    % Setup hyperref package
    \hypersetup{
      breaklinks=true,  % so long urls are correctly broken across lines
      colorlinks=true,
      urlcolor=urlcolor,
      linkcolor=linkcolor,
      citecolor=citecolor,
      }
    % Slightly bigger margins than the latex defaults
    
    \geometry{verbose,tmargin=1in,bmargin=1in,lmargin=1in,rmargin=1in}
    
    

\begin{document}
    
\begin{titlepage}
	\centering
    \vspace*{0.5 cm}
    \includegraphics[scale = 0.50]{logo.png}\\[1.0 cm]	% University Logo
	\text{\large Universidade do Minho}\\[0.1 cm]
	\text{\large Escola de Engenharia}\\[3.0 cm]
	\text{\LARGE Estruturas Criptográficas}\\[0.5 cm]				% Course Code
	\rule{\linewidth}{0.2 mm} \\[0.4 cm]
	{ \huge \bfseries Sessão Síncrona entre Emitter e Receiver}\\[0.3 cm]
	{ \LARGE Comunicação segura com e sem Curvas Elípticas }
	\rule{\linewidth}{0.2 mm} \\[3.5 cm]
	
	\begin{minipage}{0.4\textwidth}
		\begin{flushleft} \large
			\emph{\textbf{Submitted To:}}\\
			José Valença\\
            Professor Catedrático\\
             Tecnologias da Informação e Segurança\\
			\end{flushleft}
			\end{minipage}~
			\begin{minipage}{0.4\textwidth}
            
			\begin{flushright} \large
			\emph{\textbf{Submitted By :}} \\
			Diogo Araújo, A78485\\
		Diogo Nogueira, A78957\\
            Group 4\\
		\end{flushright}
        
	\end{minipage}\\[2 cm]

	
\end{titlepage}
    
   
    \newpage
	  \hypersetup{linkcolor=black}
\vfill

	\tableofcontents
\vfill

	\newpage


    \hypertarget{sessuxe3o-suxedncrona-entre-emitter-e-receiver}{%
\section{Sessão Síncrona entre Emitter e
Receiver}\label{sessuxe3o-suxedncrona-entre-emitter-e-receiver}}
\bigskip
\hypertarget{descriuxe7uxe3o-do-exercuxedcio}{%
\subsection{Descrição do
Exercício}\label{descriuxe7uxe3o-do-exercuxedcio}}

Construção de uma \textbf{sessão síncrona} de comunicação segura entre
dois agentes - \textbf{Emitter} e \textbf{Receiver}.

\textbf{Requisitos a ter em conta para a comunicação:}

\begin{itemize}
\tightlist
\item
  Gerador de \emph{nonces} (IVs)

  \begin{itemize}
  \tightlist
  \item
    \emph{Nonce} que nunca foi usado antes
  \item
    Criado aleatoriamente em cada instância da comunicação
  \end{itemize}
\item
  Cifra Simétrica \textbf{AES}

  \begin{itemize}
  \tightlist
  \item
    Autenticação de cada criptograma com \textbf{HMAC} e um modo seguro
    contra ataques aos vetores de iniciação
  \end{itemize}
\item
  Protocolo de acordo de chaves \textbf{Diffie-Hellman}

  \begin{itemize}
  \tightlist
  \item
    Com autenticação dos agentes através do esquema de assinaturas
    \textbf{DSA}
  \end{itemize}
\end{itemize}
\bigskip
\hypertarget{descriuxe7uxe3o-da-implementauxe7uxe3o}{%
\subsection{Descrição da
Implementação}\label{descriuxe7uxe3o-da-implementauxe7uxe3o}}

Perante os requisitos apresentados anteriormente teve de existir uma
decisão por parte do grupo em certos aspetos, de modo a tornar toda a
comunicação mais segura contra determinados ataques.

\textbf{Decisões tomadas:}

\begin{itemize}
\tightlist
\item
  Cifra Simétrica \textbf{AES} usará o modo \textbf{CFB}

  \begin{itemize}
  \tightlist
  \item
    Uso do modo \textbf{CFB (\emph{Cipher Feedback})} dado que ao
    usarmos este modo não estamos sujeitos a ataque mesmo que o IV em si
    seja previsível de início. Apenas é necessário garantir que este
    valor IV seja único para cada utilização. Conseguimos esta
    singularidade pela atribuição de valores aleatórios.
  \item
    Assim, IVs previsíveis são seguros no modo \textbf{CFB}, desde que
    não se repitam e que não se permita que o invasor os escolha.
  \end{itemize}
\end{itemize}
\bigskip
\hypertarget{resoluuxe7uxe3o-do-exercuxedcio}{%
\subsection{Resolução do
Exercício}\label{resoluuxe7uxe3o-do-exercuxedcio}}

    \begin{tcolorbox}[breakable, size=fbox, boxrule=1pt, pad at break*=1mm,colback=cellbackground, colframe=cellborder]
\prompt{In}{incolor}{1}{\boxspacing}
\begin{Verbatim}[commandchars=\\\{\}]
\PY{k+kn}{import} \PY{n+nn}{os}\PY{o}{,} \PY{n+nn}{io}
\PY{k+kn}{from} \PY{n+nn}{SyncPipe} \PY{k+kn}{import} \PY{n}{SyncPipe}
\PY{k+kn}{from} \PY{n+nn}{DH\PYZus{}DSA} \PY{k+kn}{import} \PY{n}{dh\PYZus{}dsa}\PY{p}{,} \PY{n}{myMAC}
\PY{k+kn}{from} \PY{n+nn}{cryptography}\PY{n+nn}{.}\PY{n+nn}{hazmat}\PY{n+nn}{.}\PY{n+nn}{primitives}\PY{n+nn}{.}\PY{n+nn}{ciphers} \PY{k+kn}{import} \PY{n}{Cipher}\PY{p}{,} \PY{n}{algorithms}\PY{p}{,} \PY{n}{modes}
\PY{k+kn}{from} \PY{n+nn}{cryptography}\PY{n+nn}{.}\PY{n+nn}{exceptions} \PY{k+kn}{import} \PY{o}{*}
\PY{k+kn}{from} \PY{n+nn}{cryptography}\PY{n+nn}{.}\PY{n+nn}{hazmat}\PY{n+nn}{.}\PY{n+nn}{backends} \PY{k+kn}{import} \PY{n}{default\PYZus{}backend}
\end{Verbatim}
\end{tcolorbox}

    \hypertarget{definiuxe7uxe3o-do-agente-emitter}{%
\subsubsection{Definição do Agente
Emitter}\label{definiuxe7uxe3o-do-agente-emitter}}

Função que trata de definir o \emph{Emitter} e seu envolvimento no
processo de comunicação segura.

\textbf{Descrição do processo:}

\begin{itemize}
\tightlist
\item
  Estabelecimento do acordo de chaves \emph{Diffie-Hellman} com
  assinatura \emph{Digital Signature Algorithm}
\item
  Criação de uma valor IV aleatório (conforme explicado na
  \textbf{Descrição da Implementação})
\item
  Criação da cifra AES no modo CFB
\item
  Criação dum HMAC para fazer em cada bloco e depois finalizar.
\end{itemize}

Com todos estes valores necessários e com um canal pronto para enviar os
dados, a ideia é ir lendo blocos de 32 \emph{bytes} devidamente
cifrados, enviando-os sucessivamente pelo canal juntamente com a
\emph{tag} HMAC criada para o efeito. Após se enviar todos os blocos de
dados fecha-se o canal.

    \begin{tcolorbox}[breakable, size=fbox, boxrule=1pt, pad at break*=1mm,colback=cellbackground, colframe=cellborder]
\prompt{In}{incolor}{2}{\boxspacing}
\begin{Verbatim}[commandchars=\\\{\}]
\PY{n}{tamanhoMensagem} \PY{o}{=} \PY{l+m+mi}{2}\PY{o}{*}\PY{o}{*}\PY{l+m+mi}{10}

\PY{k}{def} \PY{n+nf}{Emitter}\PY{p}{(}\PY{n}{connection}\PY{p}{)}\PY{p}{:}
    
    \PY{c+c1}{\PYZsh{} Acordo de chaves DH e assinatura DSA}
    \PY{n}{key} \PY{o}{=} \PY{n}{dh\PYZus{}dsa}\PY{p}{(}\PY{n}{connection}\PY{p}{)}
    
    \PY{c+c1}{\PYZsh{} Criação dum input stream para enviar a mensagem}
    \PY{n}{inputs} \PY{o}{=} \PY{n}{io}\PY{o}{.}\PY{n}{BytesIO}\PY{p}{(}\PY{n+nb}{bytes}\PY{p}{(}\PY{l+s+s1}{\PYZsq{}}\PY{l+s+s1}{D}\PY{l+s+s1}{\PYZsq{}}\PY{o}{*}\PY{n}{tamanhoMensagem}\PY{p}{,}\PY{l+s+s1}{\PYZsq{}}\PY{l+s+s1}{utf\PYZhy{}8}\PY{l+s+s1}{\PYZsq{}}\PY{p}{)}\PY{p}{)}
    
    \PY{c+c1}{\PYZsh{} Inicialização do Vector IV aleatório}
    \PY{n}{iv}  \PY{o}{=} \PY{n}{os}\PY{o}{.}\PY{n}{urandom}\PY{p}{(}\PY{l+m+mi}{16}\PY{p}{)}
    
    \PY{c+c1}{\PYZsh{} Cifra AES com o modo CFB}
    \PY{n}{cipher} \PY{o}{=} \PY{n}{Cipher}\PY{p}{(}\PY{n}{algorithms}\PY{o}{.}\PY{n}{AES}\PY{p}{(}\PY{n}{key}\PY{p}{)}\PY{p}{,} \PY{n}{modes}\PY{o}{.}\PY{n}{CFB}\PY{p}{(}\PY{n}{iv}\PY{p}{)}\PY{p}{,} 
                        \PY{n}{backend}\PY{o}{=}\PY{n}{default\PYZus{}backend}\PY{p}{(}\PY{p}{)}\PY{p}{)}\PY{o}{.}\PY{n}{encryptor}\PY{p}{(}\PY{p}{)}
    
    \PY{c+c1}{\PYZsh{} HMAC}
    \PY{n}{mac} \PY{o}{=} \PY{n}{myMAC}\PY{p}{(}\PY{n}{key}\PY{p}{)}
    
    \PY{c+c1}{\PYZsh{} Enviar o IV para o peer}
    \PY{n}{connection}\PY{o}{.}\PY{n}{send}\PY{p}{(}\PY{n}{iv}\PY{p}{)}
    
    \PY{c+c1}{\PYZsh{} Criação dum buffer para ler os blocos de 32 bytes (256 bits)}
    \PY{n}{buffer} \PY{o}{=} \PY{n+nb}{bytearray}\PY{p}{(}\PY{l+m+mi}{32}\PY{p}{)}
    
    \PY{c+c1}{\PYZsh{} lê, cifra e envia sucessivos blocos do input }
    \PY{k}{try}\PY{p}{:}
        
        \PY{c+c1}{\PYZsh{} Enquanto existem blocos (while)}
        \PY{k}{while} \PY{n}{inputs}\PY{o}{.}\PY{n}{readinto}\PY{p}{(}\PY{n}{buffer}\PY{p}{)}\PY{p}{:} 
            \PY{n}{textocifrado} \PY{o}{=} \PY{n}{cipher}\PY{o}{.}\PY{n}{update}\PY{p}{(}\PY{n+nb}{bytes}\PY{p}{(}\PY{n}{buffer}\PY{p}{)}\PY{p}{)}
            \PY{n}{mac}\PY{o}{.}\PY{n}{update}\PY{p}{(}\PY{n}{textocifrado}\PY{p}{)}
            \PY{n}{connection}\PY{o}{.}\PY{n}{send}\PY{p}{(}\PY{p}{(}\PY{n}{textocifrado}\PY{p}{,} \PY{n}{mac}\PY{o}{.}\PY{n}{copy}\PY{p}{(}\PY{p}{)}\PY{o}{.}\PY{n}{finalize}\PY{p}{(}\PY{p}{)}\PY{p}{)}\PY{p}{)}         

        \PY{c+c1}{\PYZsh{} Envia a finalização quando acontecer o último bloco}
        \PY{n}{connection}\PY{o}{.}\PY{n}{send}\PY{p}{(}\PY{p}{(}\PY{n}{cipher}\PY{o}{.}\PY{n}{finalize}\PY{p}{(}\PY{p}{)}\PY{p}{,} \PY{n}{mac}\PY{o}{.}\PY{n}{finalize}\PY{p}{(}\PY{p}{)}\PY{p}{)}\PY{p}{)}
        
    \PY{k}{except} \PY{n+ne}{Exception} \PY{k}{as} \PY{n}{err}\PY{p}{:}
        \PY{n+nb}{print}\PY{p}{(}\PY{l+s+s2}{\PYZdq{}}\PY{l+s+s2}{Erro no emissor: }\PY{l+s+si}{\PYZob{}0\PYZcb{}}\PY{l+s+s2}{\PYZdq{}}\PY{o}{.}\PY{n}{format}\PY{p}{(}\PY{n}{err}\PY{p}{)}\PY{p}{)}

    \PY{c+c1}{\PYZsh{} Fechar o stream para colocar inputs}
    \PY{n}{inputs}\PY{o}{.}\PY{n}{close}\PY{p}{(}\PY{p}{)}
    
    \PY{c+c1}{\PYZsh{} Fechar a conexão com o peer}
    \PY{n}{connection}\PY{o}{.}\PY{n}{close}\PY{p}{(}\PY{p}{)}
    
    \PY{c+c1}{\PYZsh{} Eliminar chave}
    \PY{n}{key} \PY{o}{=} \PY{k+kc}{None}
\end{Verbatim}
\end{tcolorbox}

    \hypertarget{definiuxe7uxe3o-do-agente-receiver}{%
\subsubsection{Definição do Agente
Receiver}\label{definiuxe7uxe3o-do-agente-receiver}}

Função que trata de definir o \emph{Receiver} e o seu envolvimento no
proceso de comunicação segura.

\textbf{Descrição do processo:}

\begin{itemize}
\tightlist
\item
  Estabelecimento do acordo de chaves \emph{Diffie-Hellman} com
  assinatura DSA
\item
  Recebimento do valor IV enviado pelo \emph{Emitter}
\item
  Criação da cifra AES no modo CFB
\item
  Criação do valor de MAC a verificar
\end{itemize}

O processo estipulado para o \emph{Receiver} difere do agente que lhe
envia a informação. Aqui é necessário verificar se a \emph{tag} MAC
criada é igual à que foi recebida, caso contrário estamos perante um
erro. Com as devidas verificações pode-se ir lendo os blocos que vão
surgindo por parte do \emph{Emitter} ao mesmo tempo que decifram.
Imprime-se a mensagem final no \emph{Notebook} em si.

    \begin{tcolorbox}[breakable, size=fbox, boxrule=1pt, pad at break*=1mm,colback=cellbackground, colframe=cellborder]
\prompt{In}{incolor}{3}{\boxspacing}
\begin{Verbatim}[commandchars=\\\{\}]
\PY{k}{def} \PY{n+nf}{Receiver}\PY{p}{(}\PY{n}{connection}\PY{p}{)}\PY{p}{:}
    
    \PY{c+c1}{\PYZsh{} Acordo de chaves DH e assinatura DSA}
    \PY{n}{key} \PY{o}{=} \PY{n}{dh\PYZus{}dsa}\PY{p}{(}\PY{n}{connection}\PY{p}{)}
    
    \PY{c+c1}{\PYZsh{} Inicializa um output stream para receber o texto cifrado}
    \PY{n}{outputs} \PY{o}{=} \PY{n}{io}\PY{o}{.}\PY{n}{BytesIO}\PY{p}{(}\PY{p}{)}
    
    \PY{c+c1}{\PYZsh{} Recebe o Vetor IV}
    \PY{n}{iv} \PY{o}{=} \PY{n}{connection}\PY{o}{.}\PY{n}{recv}\PY{p}{(}\PY{p}{)}
    
    \PY{c+c1}{\PYZsh{} Cifra AES com o modo CFB}
    \PY{n}{cipher} \PY{o}{=} \PY{n}{Cipher}\PY{p}{(}\PY{n}{algorithms}\PY{o}{.}\PY{n}{AES}\PY{p}{(}\PY{n}{key}\PY{p}{)}\PY{p}{,} \PY{n}{modes}\PY{o}{.}\PY{n}{CFB}\PY{p}{(}\PY{n}{iv}\PY{p}{)}\PY{p}{,} 
                        \PY{n}{backend}\PY{o}{=}\PY{n}{default\PYZus{}backend}\PY{p}{(}\PY{p}{)}\PY{p}{)}\PY{o}{.}\PY{n}{decryptor}\PY{p}{(}\PY{p}{)}
    
    \PY{c+c1}{\PYZsh{} HMAC }
    \PY{n}{mac} \PY{o}{=} \PY{n}{myMAC}\PY{p}{(}\PY{n}{key}\PY{p}{)}
    
    \PY{c+c1}{\PYZsh{} Operar a cifra: ler da conexão um bloco, autenticá\PYZhy{}lo, decifrá\PYZhy{}lo e escrever o resultado no stream de output}
    \PY{k}{try}\PY{p}{:}
        
        \PY{k}{while} \PY{k+kc}{True}\PY{p}{:}
            \PY{k}{try}\PY{p}{:}
                \PY{c+c1}{\PYZsh{} Receber do Emitter o buffer de 32 bytes e tag MAC associada}
                \PY{n}{buffer}\PY{p}{,} \PY{n}{tag} \PY{o}{=} \PY{n}{connection}\PY{o}{.}\PY{n}{recv}\PY{p}{(}\PY{p}{)}
                
                \PY{n}{ciphertext} \PY{o}{=} \PY{n+nb}{bytes}\PY{p}{(}\PY{n}{buffer}\PY{p}{)}
                \PY{n}{mac}\PY{o}{.}\PY{n}{update}\PY{p}{(}\PY{n}{ciphertext}\PY{p}{)}
                
                \PY{c+c1}{\PYZsh{} Verificação se a tag é igual à criada acima.}
                \PY{k}{if} \PY{n}{tag} \PY{o}{!=} \PY{n}{mac}\PY{o}{.}\PY{n}{copy}\PY{p}{(}\PY{p}{)}\PY{o}{.}\PY{n}{finalize}\PY{p}{(}\PY{p}{)}\PY{p}{:}
                    \PY{k}{raise} \PY{n}{InvalidSignature}\PY{p}{(}\PY{l+s+s2}{\PYZdq{}}\PY{l+s+s2}{Erro no bloco intermédio}\PY{l+s+s2}{\PYZdq{}}\PY{p}{)}
                
                \PY{c+c1}{\PYZsh{} Colocar no stream o texto decifrado e corretamente autenticado}
                \PY{n}{outputs}\PY{o}{.}\PY{n}{write}\PY{p}{(}\PY{n}{cipher}\PY{o}{.}\PY{n}{update}\PY{p}{(}\PY{n}{ciphertext}\PY{p}{)}\PY{p}{)}
                
                \PY{c+c1}{\PYZsh{} Caso já não haja mais buffer (blocos)}
                \PY{k}{if} \PY{o+ow}{not} \PY{n}{buffer}\PY{p}{:}
                    
                    \PY{c+c1}{\PYZsh{} Verificação se a tag MAC foi finalizada}
                    \PY{k}{if} \PY{n}{tag} \PY{o}{!=} \PY{n}{mac}\PY{o}{.}\PY{n}{finalize}\PY{p}{(}\PY{p}{)}\PY{p}{:}
                        \PY{k}{raise} \PY{n}{InvalidSignature}\PY{p}{(}\PY{l+s+s2}{\PYZdq{}}\PY{l+s+s2}{Erro na finalização}\PY{l+s+s2}{\PYZdq{}}\PY{p}{)}                
                    
                    \PY{c+c1}{\PYZsh{} Finalizar a cifra e escrever no stream}
                    \PY{n}{outputs}\PY{o}{.}\PY{n}{write}\PY{p}{(}\PY{n}{cipher}\PY{o}{.}\PY{n}{finalize}\PY{p}{(}\PY{p}{)}\PY{p}{)}
                    \PY{k}{break}
                    
            \PY{k}{except} \PY{n}{InvalidSignature} \PY{k}{as} \PY{n}{err}\PY{p}{:}
                \PY{k}{raise} \PY{n+ne}{Exception}\PY{p}{(}\PY{l+s+s2}{\PYZdq{}}\PY{l+s+s2}{Autenticação do ciphertext ou metadados: }\PY{l+s+si}{\PYZob{}\PYZcb{}}\PY{l+s+s2}{\PYZdq{}}\PY{o}{.}\PY{n}{format}\PY{p}{(}\PY{n}{err}\PY{p}{)}\PY{p}{)}
        
        \PY{c+c1}{\PYZsh{} Escrever no Jupyter Notebook os resultados colocados na stream}
        \PY{n+nb}{print}\PY{p}{(}\PY{n}{outputs}\PY{o}{.}\PY{n}{getvalue}\PY{p}{(}\PY{p}{)}\PY{p}{)}
        
    \PY{k}{except} \PY{n+ne}{Exception} \PY{k}{as} \PY{n}{err}\PY{p}{:}
        \PY{n+nb}{print}\PY{p}{(}\PY{l+s+s2}{\PYZdq{}}\PY{l+s+s2}{Erro no receptor: }\PY{l+s+si}{\PYZob{}0\PYZcb{}}\PY{l+s+s2}{\PYZdq{}}\PY{o}{.}\PY{n}{format}\PY{p}{(}\PY{n}{err}\PY{p}{)}\PY{p}{)}
        
    \PY{c+c1}{\PYZsh{} Fechar o stream dos outputs}
    \PY{n}{outputs}\PY{o}{.}\PY{n}{close}\PY{p}{(}\PY{p}{)}
    
    \PY{c+c1}{\PYZsh{} Fechar a conexão}
    \PY{n}{connection}\PY{o}{.}\PY{n}{close}\PY{p}{(}\PY{p}{)}
    
    \PY{c+c1}{\PYZsh{} Eliminar chave}
    \PY{n}{key} \PY{o}{=} \PY{k+kc}{None}
\end{Verbatim}
\end{tcolorbox}

    \hypertarget{criauxe7uxe3o-dos-pipes-e-execuuxe7uxe3o-dos-agentes}{%
\subsubsection{Criação dos Pipes e execução dos
Agentes}\label{criauxe7uxe3o-dos-pipes-e-execuuxe7uxe3o-dos-agentes}}

    \begin{tcolorbox}[breakable, size=fbox, boxrule=1pt, pad at break*=1mm,colback=cellbackground, colframe=cellborder]
\prompt{In}{incolor}{4}{\boxspacing}
\begin{Verbatim}[commandchars=\\\{\}]
\PY{n}{SyncPipe}\PY{p}{(}\PY{n}{Emitter}\PY{p}{,} \PY{n}{Receiver}\PY{p}{)}\PY{o}{.}\PY{n}{auto}\PY{p}{(}\PY{p}{)}
\end{Verbatim}
\end{tcolorbox}

    \begin{Verbatim}[commandchars=\\\{\}]
Está verificada a assinatura do parceiro.
Está verificada a assinatura do parceiro.
Valid DH (MAC)
Valid DH (MAC)
b'DDDDDDDDDDDDDDDDDDDDDDDDDDDDDDDDDDDDDDDDDDDDDDDDDDDDDDDDDDDDDDDDDDDDDDDDDDDDDD
DDDDDDDDDDDDDDDDDDDDDDDDDDDDDDDDDDDDDDDDDDDDDDDDDDDDDDDDDDDDDDDDDDDDDDDDDDDDDDDD
DDDDDDDDDDDDDDDDDDDDDDDDDDDDDDDDDDDDDDDDDDDDDDDDDDDDDDDDDDDDDDDDDDDDDDDDDDDDDDDD
DDDDDDDDDDDDDDDDDDDDDDDDDDDDDDDDDDDDDDDDDDDDDDDDDDDDDDDDDDDDDDDDDDDDDDDDDDDDDDDD
DDDDDDDDDDDDDDDDDDDDDDDDDDDDDDDDDDDDDDDDDDDDDDDDDDDDDDDDDDDDDDDDDDDDDDDDDDDDDDDD
DDDDDDDDDDDDDDDDDDDDDDDDDDDDDDDDDDDDDDDDDDDDDDDDDDDDDDDDDDDDDDDDDDDDDDDDDDDDDDDD
DDDDDDDDDDDDDDDDDDDDDDDDDDDDDDDDDDDDDDDDDDDDDDDDDDDDDDDDDDDDDDDDDDDDDDDDDDDDDDDD
DDDDDDDDDDDDDDDDDDDDDDDDDDDDDDDDDDDDDDDDDDDDDDDDDDDDDDDDDDDDDDDDDDDDDDDDDDDDDDDD
DDDDDDDDDDDDDDDDDDDDDDDDDDDDDDDDDDDDDDDDDDDDDDDDDDDDDDDDDDDDDDDDDDDDDDDDDDDDDDDD
DDDDDDDDDDDDDDDDDDDDDDDDDDDDDDDDDDDDDDDDDDDDDDDDDDDDDDDDDDDDDDDDDDDDDDDDDDDDDDDD
DDDDDDDDDDDDDDDDDDDDDDDDDDDDDDDDDDDDDDDDDDDDDDDDDDDDDDDDDDDDDDDDDDDDDDDDDDDDDDDD
DDDDDDDDDDDDDDDDDDDDDDDDDDDDDDDDDDDDDDDDDDDDDDDDDDDDDDDDDDDDDDDDDDDDDDDDDDDDDDDD
DDDDDDDDDDDDDDDDDDDDDDDDDDDDDDDDDDDDDDDDDDDDDDDDDDDDDDDDDDDDDDDDDD'
    \end{Verbatim}
\bigskip
    \hypertarget{observauxe7uxf5es-finais}{%
\subsection{Observações Finais}\label{observauxe7uxf5es-finais}}

\begin{itemize}
\tightlist
\item
  Fácil integração da cifra AES às entidades e o canal em si
\item
  Escolha de uma mensagem aleatoriamente criada pelo grupo para envio
  pelo canal

  \begin{itemize}
  \tightlist
  \item
    Mensagem consideravelmente grande para garantir a existência de
    blocos para uma \textbf{sessão síncrona}
  \end{itemize}
\item
  Toda a parte do acordo de chaves DH é feita num ficheiro à parte
\end{itemize}
\bigskip
    \hypertarget{referuxeancias}{%
\subsection{Referências}\label{referuxeancias}}

\begin{itemize}
\tightlist
\item
  Python Documentation, Process-based parallelism
  \url{https://docs.python.org/3/library/multiprocessing.html} (Acedido
  a 2 Março 2020)
\item
  evantotuts+, Introduction to Multiprocessing in Python
  \href{https://code.tutsplus.com/tutorials/introduction-to-multiprocessing-in-python--cms-30281}{https://code.tutsplus.com/tutorials/introduction-to-multiprocessing-in-python--cms-30281}
  (Acedido a 5 março 2020)
\item
  Wikipedia, Block ciper mode of operation
  \url{https://en.wikipedia.org/wiki/Block_cipher_mode_of_operation}
  (Acedido a 7 março 2020)
\item
  Cryptography, Symmetric
  encryption\url{https://cryptography.io/en/latest/hazmat/primitives/symmetric-encryption/}
  (Acedido a 7 março 2020)
\end{itemize}

\newpage
\section{Sessão Síncrona entre Emitter e Receiver com uso de Curvas
Elípticas}
\bigskip
\hypertarget{descriuxe7uxe3o-do-exercuxedcio}{%
\subsection{Descrição do
Exercício}\label{descriuxe7uxe3o-do-exercuxedcio}}

Versão alternativa da sessão do exercício anterior mas com o uso de
\textbf{curvas Elípticas}.

\textbf{Requisitos a ter em conta para esta versão de comunicação:}

\begin{itemize}
\tightlist
\item
  Cifra Simétrica \textbf{AES} substituída pela cifra
  \textbf{ChaCha20Poly1305}
\item
  Protocolo de acordo de chaves \textbf{Diffie-Hellman} agora por Curvas
  Elípticas (\textbf{ECDH})

  \begin{itemize}
  \tightlist
  \item
    Esquema de assinaturas \textbf{DSA} agora por Curvas Elípticas
    (\textbf{ECDSA})
  \end{itemize}
\item
  Foi escolhida a curva elíptica \emph{SECP384R1}
\end{itemize}
\bigskip
\hypertarget{descriuxe7uxe3o-da-implementauxe7uxe3o}{%
\subsection{Descrição da
Implementação}\label{descriuxe7uxe3o-da-implementauxe7uxe3o}}

Não foi necessário definir quaisquer decisões consoante o que foi
pedido. Apenas se efetuaram as substituições pedidas ficando o programa
similar ao anterior em termos de \textbf{Emitter} e \textbf{Receiver}. A
grande diferença foi a retirada da autenticação por \textbf{HMAC}, dado
que agora a cifra pedida pelo professor já é autenticada, através do
\textbf{MAC} intitulado de \textbf{Poly1305}.
\bigskip
\hypertarget{resoluuxe7uxe3o-do-exercuxedcio}{%
\subsection{Resolução do
Exercício}\label{resoluuxe7uxe3o-do-exercuxedcio}}

    \begin{tcolorbox}[breakable, size=fbox, boxrule=1pt, pad at break*=1mm,colback=cellbackground, colframe=cellborder]
\prompt{In}{incolor}{1}{\boxspacing}
\begin{Verbatim}[commandchars=\\\{\}]
\PY{k+kn}{import} \PY{n+nn}{os}\PY{o}{,} \PY{n+nn}{io}
\PY{k+kn}{from} \PY{n+nn}{SyncPipe} \PY{k+kn}{import} \PY{n}{SyncPipe}
\PY{k+kn}{from} \PY{n+nn}{ECDH\PYZus{}ECDSA} \PY{k+kn}{import} \PY{n}{ecdh\PYZus{}ecdsa}
\PY{k+kn}{from} \PY{n+nn}{cryptography}\PY{n+nn}{.}\PY{n+nn}{hazmat}\PY{n+nn}{.}\PY{n+nn}{primitives}\PY{n+nn}{.}\PY{n+nn}{ciphers}\PY{n+nn}{.}\PY{n+nn}{aead} \PY{k+kn}{import} \PY{n}{ChaCha20Poly1305}
\PY{k+kn}{from} \PY{n+nn}{cryptography}\PY{n+nn}{.}\PY{n+nn}{exceptions} \PY{k+kn}{import} \PY{o}{*}
\end{Verbatim}
\end{tcolorbox}

    \hypertarget{definiuxe7uxe3o-do-agente-emitter}{%
\subsubsection{Definição do Agente
Emitter}\label{definiuxe7uxe3o-do-agente-emitter}}

Função que trata de definir o \emph{Emitter} e seu envolvimento no
processo de comunicação segura.

\textbf{Descrição do processo:}

\begin{itemize}
\tightlist
\item
  Estabelecimento do acordo de chaves \emph{Elliptic-curve
  Diffie-Hellman} com assinatura \emph{Elliptic Curve Digital Signature
  Algorithm}
\item
  Criação de um \emph{nonce} aleatório a cada comunicação
\item
  Criação da cifra ChaCha20Poly1305
\end{itemize}

Com todos estes valores necessários e com um canal pronto para enviar os
dados, a ideia é ir lendo blocos de 32 \emph{bytes} devidamente
cifrados, enviando-os sucessivamente pelo canal. Após se enviar todos os
blocos de dados fecha-se o canal.

    \begin{tcolorbox}[breakable, size=fbox, boxrule=1pt, pad at break*=1mm,colback=cellbackground, colframe=cellborder]
\prompt{In}{incolor}{2}{\boxspacing}
\begin{Verbatim}[commandchars=\\\{\}]
\PY{n}{tamanhoMensagem} \PY{o}{=} \PY{l+m+mi}{2}\PY{o}{*}\PY{o}{*}\PY{l+m+mi}{10}

\PY{k}{def} \PY{n+nf}{Emitter}\PY{p}{(}\PY{n}{connection}\PY{p}{)}\PY{p}{:}
    
    \PY{c+c1}{\PYZsh{} Acordo de chaves DH e assinatura DSA}
    \PY{n}{key} \PY{o}{=} \PY{n}{ecdh\PYZus{}ecdsa}\PY{p}{(}\PY{n}{connection}\PY{p}{)}
        
    \PY{c+c1}{\PYZsh{} Criação dum input stream para enviar a mensagem}
    \PY{n}{inputs} \PY{o}{=} \PY{n}{io}\PY{o}{.}\PY{n}{BytesIO}\PY{p}{(}\PY{n+nb}{bytes}\PY{p}{(}\PY{l+s+s1}{\PYZsq{}}\PY{l+s+s1}{D}\PY{l+s+s1}{\PYZsq{}}\PY{o}{*}\PY{n}{tamanhoMensagem}\PY{p}{,}\PY{l+s+s1}{\PYZsq{}}\PY{l+s+s1}{utf\PYZhy{}8}\PY{l+s+s1}{\PYZsq{}}\PY{p}{)}\PY{p}{)}
    
    \PY{c+c1}{\PYZsh{} Inicialização do nonce aleatório (12 bytes)}
    \PY{n}{nonce}  \PY{o}{=} \PY{n}{os}\PY{o}{.}\PY{n}{urandom}\PY{p}{(}\PY{l+m+mi}{12}\PY{p}{)}
    
    \PY{c+c1}{\PYZsh{} Cifra ChaCha20Poly1305}
    \PY{n}{cipher} \PY{o}{=} \PY{n}{ChaCha20Poly1305}\PY{p}{(}\PY{n}{key}\PY{p}{)}
    
    \PY{c+c1}{\PYZsh{} Enviar o IV para o peer}
    \PY{n}{connection}\PY{o}{.}\PY{n}{send}\PY{p}{(}\PY{n}{nonce}\PY{p}{)}
    
    \PY{c+c1}{\PYZsh{} Criação dum buffer para ler os blocos de 32 bytes (256 bits)    }
    \PY{n}{buffer} \PY{o}{=} \PY{n+nb}{bytearray}\PY{p}{(}\PY{l+m+mi}{32}\PY{p}{)}
    
    \PY{c+c1}{\PYZsh{} lê, cifra e envia sucessivos blocos do input }
    \PY{k}{try}\PY{p}{:}
        
        \PY{c+c1}{\PYZsh{} Enquanto existem blocos (while)}
        \PY{k}{while} \PY{n}{inputs}\PY{o}{.}\PY{n}{readinto}\PY{p}{(}\PY{n}{buffer}\PY{p}{)}\PY{p}{:} 
            \PY{n}{ciphertext} \PY{o}{=} \PY{n}{cipher}\PY{o}{.}\PY{n}{encrypt}\PY{p}{(}\PY{n}{nonce}\PY{p}{,} \PY{n+nb}{bytes}\PY{p}{(}\PY{n}{buffer}\PY{p}{)}\PY{p}{,} \PY{k+kc}{None}\PY{p}{)}
            \PY{n}{connection}\PY{o}{.}\PY{n}{send}\PY{p}{(}\PY{n}{ciphertext}\PY{p}{)}       
    \PY{k}{except} \PY{n+ne}{Exception} \PY{k}{as} \PY{n}{err}\PY{p}{:}
        \PY{n+nb}{print}\PY{p}{(}\PY{l+s+s2}{\PYZdq{}}\PY{l+s+s2}{Erro no emissor: }\PY{l+s+si}{\PYZob{}0\PYZcb{}}\PY{l+s+s2}{\PYZdq{}}\PY{o}{.}\PY{n}{format}\PY{p}{(}\PY{n}{err}\PY{p}{)}\PY{p}{)}

    \PY{c+c1}{\PYZsh{} Fechar o stream para colocar inputs}
    \PY{n}{inputs}\PY{o}{.}\PY{n}{close}\PY{p}{(}\PY{p}{)}
    
    \PY{c+c1}{\PYZsh{} Fechar a conexão com o peer}
    \PY{n}{connection}\PY{o}{.}\PY{n}{close}\PY{p}{(}\PY{p}{)}
\end{Verbatim}
\end{tcolorbox}

    \hypertarget{definiuxe7uxe3o-do-agente-receiver}{%
\subsubsection{Definição do Agente
Receiver}\label{definiuxe7uxe3o-do-agente-receiver}}

Função que trata de definir o \emph{Receiver} e o seu envolvimento no
proceso de comunicação segura.

\textbf{Descrição do processo:}

\begin{itemize}
\tightlist
\item
  Estabelecimento do acordo de chaves \emph{Elliptic-curve
  Diffie-Hellman} com assinatura \emph{Elliptic Curve Digital Signature
  Algorithm}
\item
  Recebimento do \emph{nonce} enviado pelo \emph{Emitter}
\item
  Decifração com cifra ChaCha20Poly1305
\end{itemize}

O processo estipulado para o \emph{Receiver} difere do agente que lhe
envia a informação. Com as devidas verificações pode-se ir lendo os
blocos que vão surgindo por parte do \emph{Emitter} ao mesmo tempo que
decifram. Imprime-se a mensagem final no \emph{Notebook} em si.

    \begin{tcolorbox}[breakable, size=fbox, boxrule=1pt, pad at break*=1mm,colback=cellbackground, colframe=cellborder]
\prompt{In}{incolor}{3}{\boxspacing}
\begin{Verbatim}[commandchars=\\\{\}]
\PY{k}{def} \PY{n+nf}{Receiver}\PY{p}{(}\PY{n}{connection}\PY{p}{)}\PY{p}{:}
    
    \PY{c+c1}{\PYZsh{} Acordo de chaves DH e assinatura DSA}
    \PY{n}{key} \PY{o}{=} \PY{n}{ecdh\PYZus{}ecdsa}\PY{p}{(}\PY{n}{connection}\PY{p}{)}
    
    \PY{c+c1}{\PYZsh{} Inicializa um output stream para receber o texto decifrado}
    \PY{n}{outputs} \PY{o}{=} \PY{n}{io}\PY{o}{.}\PY{n}{BytesIO}\PY{p}{(}\PY{p}{)}
    
    \PY{c+c1}{\PYZsh{} Recebe o Vetor IV}
    \PY{n}{nonce} \PY{o}{=} \PY{n}{connection}\PY{o}{.}\PY{n}{recv}\PY{p}{(}\PY{p}{)}
    
    \PY{c+c1}{\PYZsh{} Cifra ChaCha20Poly1305}
    \PY{n}{cipher} \PY{o}{=} \PY{n}{ChaCha20Poly1305}\PY{p}{(}\PY{n}{key}\PY{p}{)}
    
    \PY{c+c1}{\PYZsh{} Operar a cifra: ler da conexão um bloco, autenticá\PYZhy{}lo, decifrá\PYZhy{}lo e escrever o resultado no stream de output}
    \PY{k}{try}\PY{p}{:}
        \PY{k}{while} \PY{k+kc}{True}\PY{p}{:}
            \PY{k}{try}\PY{p}{:}
                \PY{c+c1}{\PYZsh{} Receber do Emitter o buffer de 32 bytes e tag MAC associada}
                \PY{n}{buffer} \PY{o}{=} \PY{n}{connection}\PY{o}{.}\PY{n}{recv}\PY{p}{(}\PY{p}{)}

                \PY{n}{ciphertext} \PY{o}{=} \PY{n+nb}{bytes}\PY{p}{(}\PY{n}{buffer}\PY{p}{)}
                \PY{n}{decifredtext} \PY{o}{=} \PY{n}{cipher}\PY{o}{.}\PY{n}{decrypt}\PY{p}{(}\PY{n}{nonce}\PY{p}{,} \PY{n}{ciphertext}\PY{p}{,} \PY{k+kc}{None}\PY{p}{)}
                    
                \PY{c+c1}{\PYZsh{} Colocar no stream o texto decifrado e corretamente autenticado}
                \PY{n}{outputs}\PY{o}{.}\PY{n}{write}\PY{p}{(}\PY{n}{decifredtext}\PY{p}{)}
                \PY{k}{break}
            \PY{k}{except} \PY{n}{InvalidSignature} \PY{k}{as} \PY{n}{err}\PY{p}{:}
                \PY{k}{raise} \PY{n+ne}{Exception}\PY{p}{(}\PY{l+s+s2}{\PYZdq{}}\PY{l+s+s2}{Autenticação do ciphertext ou metadados: }\PY{l+s+si}{\PYZob{}\PYZcb{}}\PY{l+s+s2}{\PYZdq{}}\PY{o}{.}\PY{n}{format}\PY{p}{(}\PY{n}{err}\PY{p}{)}\PY{p}{)}
                
        \PY{c+c1}{\PYZsh{} Escrever no Jupyter Notebook os resultados colocados na stream}
        \PY{n+nb}{print}\PY{p}{(}\PY{n}{outputs}\PY{o}{.}\PY{n}{getvalue}\PY{p}{(}\PY{p}{)}\PY{p}{)}
        
    \PY{k}{except} \PY{n+ne}{Exception} \PY{k}{as} \PY{n}{err}\PY{p}{:}
        \PY{n+nb}{print}\PY{p}{(}\PY{l+s+s2}{\PYZdq{}}\PY{l+s+s2}{Erro no receptor: }\PY{l+s+si}{\PYZob{}0\PYZcb{}}\PY{l+s+s2}{\PYZdq{}}\PY{o}{.}\PY{n}{format}\PY{p}{(}\PY{n}{err}\PY{p}{)}\PY{p}{)}

    \PY{c+c1}{\PYZsh{} Fechar o stream dos outputs}
    \PY{n}{outputs}\PY{o}{.}\PY{n}{close}\PY{p}{(}\PY{p}{)}

    \PY{c+c1}{\PYZsh{} Fechar a conexão}
    \PY{n}{connection}\PY{o}{.}\PY{n}{close}\PY{p}{(}\PY{p}{)}
    
\end{Verbatim}
\end{tcolorbox}

    \hypertarget{criauxe7uxe3o-dos-pipes-e-execuuxe7uxe3o-dos-agentes}{%
\subsubsection{Criação dos Pipes e execução dos
Agentes}\label{criauxe7uxe3o-dos-pipes-e-execuuxe7uxe3o-dos-agentes}}

    \begin{tcolorbox}[breakable, size=fbox, boxrule=1pt, pad at break*=1mm,colback=cellbackground, colframe=cellborder]
\prompt{In}{incolor}{4}{\boxspacing}
\begin{Verbatim}[commandchars=\\\{\}]
\PY{n}{SyncPipe}\PY{p}{(}\PY{n}{Emitter}\PY{p}{,} \PY{n}{Receiver}\PY{p}{)}\PY{o}{.}\PY{n}{auto}\PY{p}{(}\PY{p}{)}
\end{Verbatim}
\end{tcolorbox}

    \begin{Verbatim}[commandchars=\\\{\}]
Está verificada a assinatura do parceiro.
Está verificada a assinatura do parceiro.
Valid ECDH (MAC)
Valid ECDH (MAC)
b'DDDDDDDDDDDDDDDDDDDDDDDDDDDDDDDD'
    \end{Verbatim}
\bigskip
    \hypertarget{observauxe7uxf5es-finais}{%
\subsection{Observações Finais}\label{observauxe7uxf5es-finais}}

\begin{itemize}
\tightlist
\item
  A parte do acordo de chaves ECDH com assinatura ECDSA foi feita,
  igualmente ao exercício anterior, num ficheiro à parte
\item
  Maior dificuldade em entender como usar a cifra ChaCha20Poly1305, dado
  que se trata de uma \emph{Stream Cipher} e estamos a lidar com uma
  \textbf{sessão síncrona} com uma implementação a pensar em blocos.
\end{itemize}
\bigskip
    \hypertarget{referuxeancias}{%
\subsection{Referências}\label{referuxeancias}}

\begin{itemize}
\tightlist
\item
  Wikipedia, Elliptic-curve Diffie--Hellman
  \url{https://en.wikipedia.org/wiki/Elliptic-curve_Diffie\%E2\%80\%93Hellman}
  (Acedido a 10 março 2020)
\item
  Wikipedia, Elliptic Curve Digital Signature Algorithm
  \url{https://en.wikipedia.org/wiki/Elliptic_Curve_Digital_Signature_Algorithm}
  (Acedido a 10 março 2020)
\item
  Doc Sagemath, Elliptic curves
  \url{http://doc.sagemath.org/html/en/constructions/elliptic_curves.html}
  (Acedido a 10 março 2020)
\item
  Cryptography, Authenticated encryption
  \url{https://cryptography.io/en/latest/hazmat/primitives/aead/}
  (Acedido a 10 março 2020)
\item
  Cryptography, Elliptic curve cryptography
  \url{https://cryptography.io/en/latest/hazmat/primitives/asymmetric/ec/}
  (Acedido a 11 março 2020)
\end{itemize}

    
\end{document}
